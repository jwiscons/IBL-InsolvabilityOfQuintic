% !TEX root = IBL-InsolvabilityOfQuintic.tex
\chapter{Hints\\ {\large Some possible (but not the only!) ways to get started.}}
\label{chapter:Hints}
\thispagestyle{empty}



\begin{hint*}[Theorem~\ref{thm.MonicQuadratic}]
You are solving $x^2+bx+c = 0$. Try ``completing the square'' first; then solve for $x$.
\end{hint*}

\begin{hint*}[Problem~\ref{prob.ComplexCheckin}]
Multiplying a fraction by the complex conjugate of the denominator can be an effective way to simplify an expression.
\end{hint*}

\begin{hint*}[Theorem~\ref{thm.PolarToRectangular}]
Think back to changing from polar to rectangular coordinates (or parametrizing circles or solving triangles).
\end{hint*}

\begin{hint*}[Theorem~\ref{thm.MultiplyComplex}]
Try using Theorem~\ref{thm.PolarToRectangular} $+$ trigonometric identities. 
\end{hint*}

\begin{hint*}[Lemma~\ref{lem.nthRoot1IsPowerOfZeta}]
You are working with powers (hence multiplication) so try writing $z$ in the form $z = r\cos\theta + ir\sin\theta$. Now, what does $z^n = 1$ imply about $r$ and $\theta$?
\end{hint*}

\begin{hint*}[Theorem~\ref{thm.nthRoots1}]
You may want to view this as the following ``if and only if'' statement: $z$ is an $n^\text{th}$ root of $1$ $\iff$ $z = (\zeta_n)^k$ for some $0\le k\le n-1$. Now make use of the previous lemma and theorems you proved. Don't forget to explain why each of $1, \zeta_n, (\zeta_n)^2, \ldots, (\zeta_n)^{n-1}$ are all different.
\end{hint*}

\begin{hint*}[Theorem~\ref{thm.RootsRealCoeff}]
Suppose that $z$ is a root of $p(x)$. Then $p(z) = 0$, so  $a_nz^n + a_{n-1}z^{n-1} +\cdots+a_2z^2+a_1z+a_0 = 0$. This last equation is is just comparing two complex numbers---try taking the conjugate of both sides. Fact~\ref{fact.ComplexLaws} is helpful.
\end{hint*}





