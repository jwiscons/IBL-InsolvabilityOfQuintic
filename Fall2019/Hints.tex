% !TEX root = IBL-InsolvabilityOfQuintic.tex
\chapter{Hints}
\label{chapter:Hints}
\thispagestyle{empty}

Below are some hints, which should be interpreted as possible (but not the only!) ways to get started.

\begin{hint*}[Theorem~\ref{thm.MonicQuadratic}]
You are solving $x^2+bx+c = 0$. Try ``completing the square'' first; then solve for $x$.
\end{hint*}

\begin{hint*}[Problem~\ref{prob.ComplexCheckin}]
Multiplying a fraction by the complex conjugate of the denominator can be an effective way to simplify an expression.
\end{hint*}

\begin{hint*}[Theorem~\ref{thm.PolarToRectangular}]
Think back to changing from polar to rectangular coordinates (or parametrizing circles or solving triangles).
\end{hint*}

\begin{hint*}[Theorem~\ref{thm.MultiplyComplex}]
Try using Theorem~\ref{thm.PolarToRectangular} $+$ trigonometric identities. 
\end{hint*}

\begin{hint*}[Problem~\ref{prob.nthRoots}]
You want to find a $z$ such that $z^4 = \zeta_3$. You are working with powers (hence multiplication), so try writing $z$ in the form $z = r\cos\theta + ir\sin\theta$. Now you can use Corollary~\ref{cor.DeMoivre} to simplify $z^4$ and compare with $\zeta_3$. What can you deduce about $r$ and $\theta$?
\end{hint*}

\begin{hint*}[Lemma~\ref{lem.nthRoot1IsPowerOfZeta}]
Similar to Problem~\ref{prob.nthRoots}, try writing $z$ in the form $z = r\cos\theta + ir\sin\theta$. Now, what does $z^n = 1$ imply about $r$ and $\theta$?
\end{hint*}

\begin{hint*}[Lemma~\ref{lem.ReducePowerOfZeta}]
It may be helpful to draw some pictures first. Try plotting $\zeta_8$, $(\zeta_8)^2$, $(\zeta_8)^3$, \ldots, $(\zeta_8)^8$, $(\zeta_8)^{14}$, $(\zeta_8)^{85}$. Now, you know by a previous problem that $(\zeta_n)^n = 1$, so also $(\zeta_n)^{2n} = 1$ and so on. Try (using the division algorithm) to write $k = qn +r$ for some $q,r\in \mathbb{Z}$ with $0\le r \le n-1$ and plug that into $(\zeta_n)^{k}$.
\end{hint*}


\begin{hint*}[Theorem~\ref{thm.nthRoots1}]
You may want to view this as the following ``if and only if'' statement: $z$ is an $n^\text{th}$ root of $1$ $\iff$ $z = (\zeta_n)^k$ for some $0\le k\le n-1$. Now make use of the previous lemma and theorems you proved. Don't forget to explain why each of $1, \zeta_n, (\zeta_n)^2, \ldots, (\zeta_n)^{n-1}$ are all different.
\end{hint*}

\begin{hint*}[Theorem~\ref{thm.RootsRealCoeff}]
Suppose that $z$ is a root of $p(x)$. Then $p(z) = 0$, so  $a_nz^n + a_{n-1}z^{n-1} +\cdots+a_2z^2+a_1z+a_0 = 0$. This last equation is is just comparing two complex numbers---try taking the conjugate of both sides. Fact~\ref{fact.ComplexLaws} is helpful.
\end{hint*}

\begin{hint*}[Problem~\ref{prob.QAdjoinRoot5Inverse}]
You are trying to find $(a+b\sqrt{5})^{-1} = \frac{1}{a+b\sqrt{5}}$. Try multiplying top and bottom by the conjugate: $a-b\sqrt{5}$.
\end{hint*}

\begin{hint*}[Theorem~\ref{thm.BasicFieldProps}]
For the first part, notice that $x\cdot0 = x(0+0)$. For the last part, remember that the definition of a field ensures that $F$ has at least two elements, so there is some $a\in F$ with $a\neq 0$. Now, what happens if $0=1$?
\end{hint*}

\begin{hint*}[Theorem~\ref{thm.ZpField}]
The crux is to show that every nonzero element has a multiplicative inverse when $n$ is prime. Let $a\in (\mathbb{Z}_n)^*$. You need to find some integer $b$ such that $ab=1$ modulo $p$. Now, since $a\in (\mathbb{Z}_n)^*$ and  $n$ is prime, $\gcd(a,p) = 1$. By B\'ezout's Lemma, there exist $k,l\in \mathbb{Z}$ such that $1 = ka+lp$. What happens when you consider the equation $1 = ka+lp$ modulo $p$?
\end{hint*}

\begin{hint*}[Problem~\ref{prob.SubfieldRoot2PlusI}]
If $T_3$ is a subfield, then, in particular, it is closed under multiplication, so it must be that $\alpha^2\in T_3$. What does this imply?
\end{hint*}









