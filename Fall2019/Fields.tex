\chapter{Fields}
\label{chapter:Fields}
\thispagestyle{empty}


In Chapter~\ref{chapter:PolyEquations}, we explored problems about finding and expressing roots of polynomials, finally arriving at the goal of the course: proving that there are quintic polynomials that are \emph{not} solvable by radicals. This chapter serves two main purposes. First, as we look at roots of polynomials and how they can be expressed, it will be convenient to have a common world (i.e.~number system) in which they live. For us, this will be the complex numbers, denoted $\mathbb{C}$, which will be reviewed below. Our work with complex numbers will also supply the necessary language to properly talk about $n^{\text{th}}$-roots. Second, we are still in need of a proper definition of what it means for a polynomial to be ``solvable by radicals''; this is where the chapter will finish. But the middle of the chapter is perhaps the most interesting. There, on the way to defining ``solvable by radicals'', we will be led to abstract the structure of $\mathbb{C}$ (and of $\mathbb{Q}$ and $\mathbb{R}$), arriving at the definition of a \emph{field}. 

\begin{section}{Complex Numbers}


\end{section}




Next stop complex numbers and hamiltonians







