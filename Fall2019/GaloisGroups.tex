% !TEX root = IBL-InsolvabilityOfQuintic.tex
\chapter{Galois Groups}
\label{chapter:GaloisGroups}
\thispagestyle{empty}

We finished Chapter~\ref{chapter:AlgebraicExtensions} by computing automorphism groups of field extensions. We also began to connect the subfields of a field $L$ that extends $F$ to subgroups of the automorphism group of $L$ over $F$. In this section, we narrow our focus on which types of extension fields we consider, and in doing so, we significantly sharpen what we can say about this connection. It will be lynchpin of our argument showing that not all polynomials over $\mathbb{Q}$ are solvable by radicals over $\mathbb{Q}$.

Also, from here on, we will exclusively focus on subfields of $\mathbb{C}$. This will streamline (and simplify) our work, but it will also obscure just how general the theory is. Which is to say, this is more the beginning of the story than the end. 

% % % % % % % % % % % % % % % % % % % % % % % % % % % % % % % % % % % % % % % % % % % %
% % % % % % % % % % % % % % % % % % % % % % % % % % % % % % % % % % % % % % % % % % % %
% SECTION
% % % % % % % % % % % % % % % % % % % % % % % % % % % % % % % % % % % % % % % % % % % %
% % % % % % % % % % % % % % % % % % % % % % % % % % % % % % % % % % % % % % % % % % % %
\section{Galois groups of polynomials}

\begin{definition}\label{def.SplittingField}
Let $F$ be a subfield of $\mathbb{C}$, and let $p(x) \in F[x]$. We define $F^{p(x)}$ to be the subfield of $\mathbb{C}$ generated by $F$ and all roots of $p(x)$.
That is, if $r_1,\ldots,r_n$ are the roots of $p(x)$ in $\mathbb{C}$, then $F^{p(x)}=F(r_1,\ldots,r_n)$. 
\end{definition}

For example, if  $p(x) = x^5 - 1$, then by Theorem~\ref{thm.nthRoots1}, the roots of $p(x)$ are $1,\zeta_5,\zeta_5^2,\zeta_5^3,\zeta_5^4$, so $\mathbb{Q}^{p(x)}=\mathbb{Q}(1,\zeta_5,\zeta_5^2,\zeta_5^3,\zeta_5^4)$.

\begin{problem}
Let $p(x) = x^5 - 1$. Use Theorem~\ref{thm.FieldAdjoinElementsContainedInField} to explain why $\mathbb{Q}^{p(x)}=\mathbb{Q}(\zeta_5)$.
\end{problem}

\begin{problem}
Let $p(x) = x^3 - 2$. Explain why $\mathbb{Q}^{p(x)}\neq\mathbb{Q}(\sqrt[3]{2})$.
\end{problem}

\begin{problem}
For each field $F$ below, find a polynomial $p(x)\in \mathbb{Q}[x]$ such that $F=\mathbb{Q}^{p(x)}$.
\begin{multicols}{2}
\begin{enumerate}
\item $F=\mathbb{Q}(\sqrt{2})$
\item $F=\mathbb{Q}(\sqrt{2},i)$
\item $F=\mathbb{Q}(\sqrt[3]{11},\zeta_3)$
\item $F=\mathbb{Q}(\zeta_n)$
\end{enumerate}
\end{multicols}
\end{problem}

time to define galois groups!!


\begin{example}
Let's determine $\Aut(\mathbb{Q}(\sqrt[3]{11},\zeta_3)/\mathbb{Q})$ by first finding a good description of the elements in $\mathbb{Q}(\sqrt[3]{11},\zeta_3)$. We'll use Theorem~\ref{thm.BasisChainExtensionField}. Now, we know that $x^3-11$ is irreducible by the EIC, so it is the minimal polynomial of $\sqrt[3]{11}$ of $\mathbb{Q}$.  Also, $\zeta_3$ is a root of $x^2+x+1$, which is irreducible over $\mathbb{Q}(\sqrt[3]{11}$ by Theorem~\ref{thm.ReducibilityTestDegree2or3} (since the roots of $x^2+x+1$ are not real numbers, hence not in $\mathbb{Q}(\sqrt[3]{11}$). Thus, $\sqrt[3]{11}$ has degree $3$ over $\mathbb{Q}$ and $\zeta_3$ has degree $2$ over $\mathbb{Q}(\sqrt[3]{11}$. Using Theorems~\ref{thm.BasisExtensionField} and~\ref{thm.BasisChainExtensionField}, we find that $1,\sqrt[3]{11},\sqrt[3]{11}^2,\zeta_3,\sqrt[3]{11}\zeta_3,\sqrt[3]{11}^2\zeta_3$ is a basis for $\mathbb{Q}(\sqrt[3]{11},\zeta_3)$ over $\mathbb{Q}$, so 
\[\mathbb{Q}(\sqrt[3]{11},\zeta_3) = a+b\sqrt[3]{11}+c\sqrt[3]{11}^2+d\zeta_3+e\sqrt[3]{11}\zeta_3+f\sqrt[3]{11}^2\zeta_3\mid a,b,c,d,e,f\in\mathbb{Q}\}.\]
As before, every function $\phi\in \Aut(\sqrt[3]{11},\zeta_3)/\mathbb{Q})$ is determined by a formula of the form 
\begin{align*}
\phi(a+ &b\sqrt[3]{11}+  c\sqrt[3]{11}^2+d\zeta_3+e\sqrt[3]{11}\zeta_3+f\sqrt[3]{11}^2\zeta_3) \\
& = \phi(a)+\phi(b\sqrt[3]{11})+\phi(c\sqrt[3]{11}^2)+\phi(d\zeta_3)+\phi(e\sqrt[3]{11}\zeta_3)+\phi(f\sqrt[3]{11}^2\zeta_3)\\
& =\phi(a)+\phi(b)\phi(\sqrt[3]{11})+\phi(c)\phi(\sqrt[3]{11})^2+\phi(d)\phi(\zeta_3)+\phi(e)\phi(\sqrt[3]{11})\phi(\zeta_3)+\phi(f)\phi(\sqrt[3]{11})^2\phi(\zeta_3)\\
& =a+b\phi(\sqrt[3]{11})+c\phi(\sqrt[3]{11})^2+d\phi(\zeta_3)+e\phi(\sqrt[3]{11})\phi(\zeta_3)+f\phi(\sqrt[3]{11})^2\phi(\zeta_3),
\end{align*}
so again, we just need to decide the value for $\phi(\sqrt[3]{11})$ and $\phi(\zeta_3)$. By Theorem~\ref{thm.HomFixingFPermutesRootsOfPolysOverF}, $\phi(\sqrt[3]{11})$ must be another root of $x^3-11$ and $\phi(\zeta_3)$ must be another root of $x^2+x+1$. As there are 3 possibilities for $\phi(\sqrt[3]{11})$ and 2 possibilities for $\phi(\zeta_3)$, we have 6 possible maps in total. Let's organize the possibilities with a table.
\begin{flushleft}
\tabulinesep = 2mm
\begin{tabu}  {X[$c,m]|[2pt]X[$c,m]|X[$c,m]|X[$c,m]|X[$c,m]|X[$c,m]|X[$c,m]}
 & \phi_1 & \phi_2 & \phi_3 & \phi_4 & \phi_5 & \phi_6 \\ \tabucline[2pt]{-}
\sqrt[3]{11} \mapsto  & \sqrt[3]{11} & \sqrt[3]{11}\zeta_3 & \sqrt[3]{11}(\zeta_3)^2 & \sqrt[3]{11} & \sqrt[3]{11}\zeta_3 & \sqrt[3]{11}(\zeta_3)^2\\  \hline 
\zeta_3 \mapsto  & \zeta_3  & \zeta_3  & \zeta_3  &(\zeta_3)^2  & (\zeta_3)^2  & (\zeta_3)^2 \\ \hline 
\end{tabu}
\end{flushleft}
To check if the maps are actually in $\Aut(\mathbb{Q}(\sqrt[3]{11},\zeta_3)/\mathbb{Q})$, we again use Fact~\ref{fact.SameMinPolyIsomorphic}. Applying the fact to $\mathbb{Q}(\sqrt[3]{11})$ over $\mathbb{Q}$ and then $\mathbb{Q}(\sqrt[3]{11},\zeta_3)$ over $\mathbb{Q}(\sqrt[3]{11})$, we find that every map is an isomorphism that fixes $\mathbb{Q}$, and we just need to check that the image is $\mathbb{Q}(\sqrt[3]{11},\zeta_3)$ (and not some different field). The images of the maps fields of the form $\mathbb{Q}(\sqrt[3]{11}\zeta_3^m,\zeta_3^n)$, and using Theorem~\ref{thm.FieldAdjoinElementsContainedInField}  


\end{example}