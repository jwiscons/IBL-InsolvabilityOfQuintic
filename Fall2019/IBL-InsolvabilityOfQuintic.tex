\documentclass[12pt,oneside]{book}

\usepackage[scale=2]{ccicons}
\usepackage{enumitem}
\usepackage{multicol}
\usepackage[labelsep=period]{caption}
\usepackage[labelformat=simple,labelfont={}]{subcaption}
\usepackage{makecell}
\usepackage{tabu}
\usepackage[table]{xcolor}
\usepackage{tikz}
\usetikzlibrary{arrows,automata,positioning}
\usepackage{rotating}
\usepackage[notextcomp]{kpfonts} 
\usepackage{graphicx}
\usepackage{eurosym}
\usepackage{amsfonts}
\usepackage{amsmath}
\usepackage{amssymb}
\usepackage{stmaryrd}
\usepackage{wasysym}
\usepackage{amsthm}
\usepackage[margin=1in]{geometry}
\usepackage[hang,flushmargin,symbol*]{footmisc}
\usepackage{color}
\definecolor{darkblue}{rgb}{0, 0, .6}
\definecolor{grey}{rgb}{.7, .7, .7}
\usepackage[breaklinks]{hyperref}
\hypersetup{
	colorlinks=true,
	linkcolor=darkblue,
	anchorcolor=darkblue,
	citecolor=darkblue,
	pagecolor=darkblue,
	urlcolor=darkblue,
	pdftitle={},
	pdfauthor={}
}

\usepackage{fancyhdr}
\pagestyle{fancy}
\lhead{\leftmark}
\chead{}
\rhead{}
\lfoot{}
\cfoot{\thepage}
\rfoot{}

\theoremstyle{definition}
\newtheorem{theorem}{Theorem}[chapter]
\newtheorem{acknowledgement}[theorem]{Acknowledgement}
\newtheorem{algorithm}[theorem]{Algorithm}
\newtheorem{axiom}[theorem]{Axiom}
\newtheorem{case}[theorem]{Case}
\newtheorem{claim}[theorem]{Claim}
\newtheorem{conclusion}[theorem]{Conclusion}
\newtheorem{condition}[theorem]{Condition}
\newtheorem{conjecture}[theorem]{Conjecture}
\newtheorem{corollary}[theorem]{Corollary}
\newtheorem{criterion}[theorem]{Criterion}
\newtheorem{definition}[theorem]{Definition}
\newtheorem{example}[theorem]{Example}
\newtheorem{exercise}[theorem]{Exercise}
\newtheorem{fact}[theorem]{Fact}
\newtheorem{journal}[theorem]{Journal}
\newtheorem{lemma}[theorem]{Lemma}
\newtheorem{notation}[theorem]{Notation}
\newtheorem{problem}[theorem]{Problem}
\newtheorem{proposition}[theorem]{Proposition}
\newtheorem{remark}[theorem]{Remark}
\newtheorem{solution}[theorem]{Solution}
\newtheorem{summary}[theorem]{Summary}
\newtheorem{skeleton}[theorem]{Skeleton Proof}
\newtheorem{activity}[theorem]{Activity}
\newtheorem{intuitivedef}[theorem]{Intuitive Definition}

\newtheorem*{hint*}{Hint}

\newsavebox{\savepar}
\newenvironment{textbox}{\noindent\begin{lrbox}{\savepar}\begin{minipage}[c]{.98\textwidth}}{\end{minipage}\end{lrbox}\fcolorbox{black}{white}{\usebox{\savepar}}}

\newcommand{\dom}{\operatorname{Dom}}
\newcommand{\codom}{\operatorname{Codom}}
\newcommand{\range}{\operatorname{Rng}}
\newcommand{\Spin}{\operatorname{Spin}}
\newcommand{\lcm}{\operatorname{lcm}}

\renewcommand\thesubfigure{(\alph{subfigure})}

\setenumerate[1]{label=\rm{(\arabic*)}}

\begin{document}

\title{Insolvability of the Quintic\\ {\small \normalfont An unofficial sequel to \href{https://github.com/dcernst/IBL-AbstractAlgebra}{An Inquiry-Based Approach to Abstract Algebra} by \href{http://danaernst.com}{Dana C. Ernst}}}
\author{Joshua Wiscons\\
California State University, Sacramento}
\date{Spring 2019}

\maketitle

\noindent\copyright{ \the\year\ Joshua Wiscons.  Some Rights Reserved.\\

\bigskip

\noindent The most up-to-date version of these notes on can be found on GitHub:
\begin{center}
\url{https://github.com/jwiscons/IBL-InsolvabilityOfQuintic}
\end{center}

\bigskip

\noindent This work is licensed under the Creative Commons Attribution-Share Alike 4.0 United States License.  You may copy, distribute, display, and perform this copyrighted work, but only if you give credit to Joshua Wiscons, and all derivative works based upon it must be published under the Creative Commons Attribution-Share Alike 4.0 International License. Please attribute this work to Joshua Wiscons, Mathematics Faculty at California State University, Sacramento, \url{joshua.wiscons@csus.edu}. To view a copy of this license, visit
\begin{center}
\url{https://creativecommons.org/licenses/by-sa/4.0/}
\end{center}
or send a letter to Creative Commons, 171 Second Street, Suite 300, San Francisco, California, 94105, USA.}

\medskip

\begin{center}
\ccbysa
\end{center}

\noindent This work is designed to extend \href{https://github.com/dcernst/IBL-AbstractAlgebra}{An Inquiry-Based Approach to Abstract Algebra} by \href{http://danaernst.com}{Dana C. Ernst}. The presentation of the material is heavily influenced by the book \href{https://www.amazon.com/Abstract-Algebra-Introduction-Robert-Redfield/dp/020143721X}{Abstract Algebra: A Concrete Introduction} by \href{https://www.hamilton.edu/academics/our-faculty/directory/faculty-detail/robert-redfield}{Robert H. Redfield}. Many thanks to both Dana and Bob!

\tableofcontents

\chapter{Solving polynomial equations and the main question}
\label{chapter:PolyEquations}
\thispagestyle{empty}

The story begins. Can you count all of the times in your career that you've had to find the zeros of a quadratic polynomial? What about a cubic polynomial? A quartic? Likely your answers are decreasing rapidly, and it's also likely that you have only solved cubic and quartic equations in very special situations. Why? Is it just that cubic and quartic equations are difficult to solve or could it be that some are impossible to ``solve.'' 

People have been investigating how to solve polynomial equations for about 4000 years. Let's get started.

\begin{problem}\label{prob.IntroFindZeros}
Determine the roots (i.e.~zeros) of each of the following. Try to use tools you've accumulated over the years, but you may well need a computer program (e.g. \href{https://www.wolframalpha.com}{WolframAlpha}) for some of them. For each problem, make a note about \emph{how} you found the roots. Try for exact answers, but you can approximate if needed.
\begin{multicols}{2}
\begin{enumerate}
\item $p(x) = x^2 - 5x + 6$
\item $q(x) = (x-3)^2 - 2$
\item $r(x) = x^2 + x + 1$
\item $s(x) = x^3-3x-2$ %rational root theorem
\item $f(x) = x^3 - 5$
\item $g(x) = x^4 - 1$
\item $a(x) = x^5 - 1$
\item $b(x) = x^5 +5 x^4-5$
\end{enumerate}
\end{multicols}
\end{problem}

\begin{problem}\label{prob.IntroFindZerosWhere}
For each part of Problem~\ref{prob.IntroFindZeros}, write down the ``smallest'' number system needed to express the roots of the given polynomial. Possible answers might be $\mathbb{Z}$ (integers), $\mathbb{Q}$ (rational numbers), $\mathbb{Z}$ together with $\sqrt{3}$, $\mathbb{Q}$ together with $\sqrt{-1}$ and $\sqrt[3]{5}$, etc.
\end{problem}

% % % % % % % % % % % % % % % % % % % % % % % % % % % % % % % % % % % % % % % % % % % %
% % % % % % % % % % % % % % % % % % % % % % % % % % % % % % % % % % % % % % % % % % % %
% SECTION
% % % % % % % % % % % % % % % % % % % % % % % % % % % % % % % % % % % % % % % % % % % %
% % % % % % % % % % % % % % % % % % % % % % % % % % % % % % % % % % % % % % % % % % % %
\begin{section}{Solving polynomial equations with formulas}
Though you may have solved the first three parts of Problem~\ref{prob.IntroFindZeros} different ways, there was one tool that would have solved them all: the quadratic formula. It will be valuable to (re)discover why it's true. 

First, let's slightly simplify things. Notice that $\alpha$ is a root of $ax^2 + bx + c$ if and only if $\alpha$ is a root of $x^2 + \frac{b}{a}x + \frac{c}{a}$ (assuming $a\neq 0$). This means that an arbitrary quadratic polynomial can always be converted to a quadratic polynomial whose leading coefficient is $1$ and in such a way that they have the same roots. Thus, if we have a formula that finds the roots of so-called \emph{monic} quadratic polynomials, we can actually use it to find the roots of \emph{all} quadratic polynomials.

\begin{definition}
A polynomial whose leading coefficient is $1$ is called a $\textbf{monic}$ polynomial.
\end{definition}

Now, let's (re)derive the quadratic formula for monic polynomials. Remember, we are \emph{(re)deriving} it, so \emph{please don't use the quadratic formula in your proof of the next theorem.}

\begin{theorem}\label{thm.MonicQuadratic}
The roots of $p(x) = x^2 + bx + c$ are \[\frac{-b\pm\sqrt{b^2 - 4c}}{2}.\]
\end{theorem}

\begin{problem}
Describe a number system, as small as possible, that is capable of expressing the roots of \emph{any} quadratic polynomial whose \emph{coefficients are all rational numbers}.
\end{problem}

Well, that takes care of quadratic polynomials. What about finding the roots of cubic polynomials? Well, it turns out that there is indeed a cubic formula, though it's decidedly more complicated than the quadratic formula. 

A method for deriving a cubic formula, due to \href{https://en.wikipedia.org/wiki/Scipione_del_Ferro}{Scipione del Ferro} and \href{https://en.wikipedia.org/wiki/Niccol�_Fontana_Tartaglia}{Tartaglia}, was published in a book by \href{https://en.wikipedia.org/wiki/Gerolamo_Cardano}{Cardano} in 1545. The starting point is to take a general cubic polynomial and first convert it to a monic polynomial (as we did above) and then convert it to a cubic of the form $x^3 + px +q$, always with the same roots as the original. Then, with work, one arrives at the following formula. 

\begin{fact}\label{fact:Cardano}
The roots of $p(x) = x^3 + px + q$ are \[\alpha + \beta, \left(-\frac{1}{2} + \frac{\sqrt{-3}}{2}\right)\alpha + \left(-\frac{1}{2} - \frac{\sqrt{-3}}{2}\right)\beta, \left(-\frac{1}{2} - \frac{\sqrt{-3}}{2}\right)\alpha + \left(-\frac{1}{2} + \frac{\sqrt{-3}}{2}\right)\beta\]
where $\alpha = \sqrt[3]{-\frac{q}{2} + \sqrt{\frac{q^2}{4}+\frac{p^3}{27}}}$ and $\beta = \sqrt[3]{-\frac{q}{2} - \sqrt{\frac{q^2}{4}+\frac{p^3}{27}}}$
\end{fact}

\begin{problem}
Describe a number system, as small as possible, that is capable of expressing the roots of \emph{any} cubic polynomial of the form $x^3 + px + q$ where $p,q\in \mathbb{Q}$.
\end{problem}

\begin{problem}
Use Fact~\ref{fact:Cardano} to write out the roots of $p(x) = x^3-2x-4$. Are any of the roots integers? Check your answer with \href{https://www.wolframalpha.com}{WolframAlpha}. Does this expose any issues with using the formula?
\end{problem}

For more details on solving cubic equations, you can use start with the \href{https://en.wikipedia.org/wiki/Cubic_function#Derivation_of_the_roots}{Wikipedia page about cubic functions}. And now\ldots quartic functions? Perhaps you have a guess.

\begin{problem}
Use the internet and/or library to determine if there is a quartic formula, i.e.~a formula to find the roots of fourth-degree polynomials. If there is a quartic formula, who are some people that discovered methods to derive it and when did they discover their method?
\end{problem}
\end{section}

% % % % % % % % % % % % % % % % % % % % % % % % % % % % % % % % % % % % % % % % % % % %
% % % % % % % % % % % % % % % % % % % % % % % % % % % % % % % % % % % % % % % % % % % %
% SECTION
% % % % % % % % % % % % % % % % % % % % % % % % % % % % % % % % % % % % % % % % % % % %
% % % % % % % % % % % % % % % % % % % % % % % % % % % % % % % % % % % % % % % % % % % %
\begin{section}{The main question(s)}
We now turn our attention to finding the roots of polynomials of degree five and higher. Well, what do you think: is there a quintic formula? Actually, there are two questions here: (1) what do we mean by ``formula'' (e.g.~what symbols/functions can we use), and (2) whatever we mean by formula, is there one that works for \emph{all} quintic polynomials?

You investigated two particular quintics in Problem~\ref{prob.IntroFindZeros}---what did you find? Did you find exact expressions for the roots? If so, \emph{how} were the roots expressed; that is, what symbols/functions were needed to write out the roots? 

Let's first try to tackle the ``what do we mean by formula'' question. Guided by the quadratic and cubic formulas, let's agree that we are looking for a formula that expresses the roots of an arbitrary quintic polynomial in terms of the coefficients of the polynomial using just the operations of addition, subtraction, multiplication, division, and the extraction of roots (square roots, cube roots,\ldots). This leads to the following intuitive definition, that we will work to sharpen later.

\begin{intuitivedef}
A polynomial is said to be \textbf{solvable by radicals} if every root of the polynomial can be expressed in terms of the coefficients of the polynomial using the operations of addition, subtraction, multiplication, division, and $\sqrt[n]{\phantom{x}}$ for any positive integer $n$.
\end{intuitivedef}

\begin{problem}
Find the roots of $p(x) = x^4 - 2x^2 -1$. Explain why $p(x)$ is solvable by radicals.
\end{problem}

\begin{problem}
Explain why every quadratic polynomial is solvable by radicals.
\end{problem}

Returning to quintic polynomials, our main question is as follows.

\begin{mainquestion} 
Does there exist a ``quintic formula'' that expresses the roots of an arbitrary quintic polynomial, in terms of the coefficients of the polynomial, using just rational numbers together with the operations of addition, subtraction, multiplication, division, and the extraction of roots?
\end{mainquestion}

Well, if the answer is yes, then it must be that every quintic polynomial is solvable by radicals. \textit{Spoiler Alert!}  Our goal (for the rest of the book!) is to prove the following theorem, in a rather elegant way. 

\begin{maintheorem}
Not every quintic polynomial is solvable by radicals.
\end{maintheorem}

\begin{maincorollary}
There is \emph{no} ``quintic formula'' that expresses the roots of an arbitrary quintic polynomial in terms of the coefficients of the polynomial using just the operations of addition, subtraction, multiplication, division, and the extraction of roots.
\end{maincorollary}

\begin{center}
Boom!
\end{center}

\end{section}









\appendix
% !TEX root = IBL-InsolvabilityOfQuintic.tex
\chapter{Hints}
\label{chapter:Hints}
\thispagestyle{empty}

Below are some hints, which should be interpreted as possible (but not the only!) ways to get started.

\begin{hint*}[Theorem~\ref{thm.MonicQuadratic}]
You are solving $x^2+bx+c = 0$. Try ``completing the square'' first; then solve for $x$.
\end{hint*}

\begin{hint*}[Problem~\ref{prob.ComplexCheckin}]
Multiplying a fraction by the complex conjugate of the denominator can be an effective way to simplify an expression.
\end{hint*}

\begin{hint*}[Theorem~\ref{thm.PolarToRectangular}]
Think back to changing from polar to rectangular coordinates (or parametrizing circles or solving triangles).
\end{hint*}

\begin{hint*}[Theorem~\ref{thm.MultiplyComplex}]
Try using Theorem~\ref{thm.PolarToRectangular} $+$ trigonometric identities. 
\end{hint*}

\begin{hint*}[Problem~\ref{prob.nthRoots}]
You want to find a $z$ such that $z^4 = \zeta_3$. You are working with powers (hence multiplication), so try writing $z$ in the form $z = r\cos\theta + ir\sin\theta$. Now you can use Corollary~\ref{cor.DeMoivre} to simplify $z^4$ and compare with $\zeta_3$. What can you deduce about $r$ and $\theta$?
\end{hint*}

\begin{hint*}[Lemma~\ref{lem.nthRoot1IsPowerOfZeta}]
Similar to Problem~\ref{prob.nthRoots}, try writing $z$ in the form $z = r\cos\theta + ir\sin\theta$. Now, what does $z^n = 1$ imply about $r$ and $\theta$?
\end{hint*}

\begin{hint*}[Lemma~\ref{lem.ReducePowerOfZeta}]
It may be helpful to draw some pictures first. Try plotting $\zeta_8$, $(\zeta_8)^2$, $(\zeta_8)^3$, \ldots, $(\zeta_8)^8$, $(\zeta_8)^{14}$, $(\zeta_8)^{85}$. Now, you know by a previous problem that $(\zeta_n)^n = 1$, so also $(\zeta_n)^{2n} = 1$ and so on. Try (using the division algorithm) to write $k = qn +r$ for some $q,r\in \mathbb{Z}$ with $0\le r \le n-1$ and plug that into $(\zeta_n)^{k}$.
\end{hint*}


\begin{hint*}[Theorem~\ref{thm.nthRoots1}]
You may want to view this as the following ``if and only if'' statement: $z$ is an $n^\text{th}$ root of $1$ $\iff$ $z = (\zeta_n)^k$ for some $0\le k\le n-1$. Now make use of the previous lemma and theorems you proved. Don't forget to explain why each of $1, \zeta_n, (\zeta_n)^2, \ldots, (\zeta_n)^{n-1}$ are all different.
\end{hint*}

\begin{hint*}[Theorem~\ref{thm.RootsRealCoeff}]
Suppose that $z$ is a root of $p(x)$. Then $p(z) = 0$, so  $a_nz^n + a_{n-1}z^{n-1} +\cdots+a_2z^2+a_1z+a_0 = 0$. This last equation is is just comparing two complex numbers---try taking the conjugate of both sides. Fact~\ref{fact.ComplexLaws} is helpful.
\end{hint*}

\begin{hint*}[Problem~\ref{prob.QAdjoinRoot5Inverse}]
You are trying to find $(a+b\sqrt{5})^{-1} = \frac{1}{a+b\sqrt{5}}$. Try multiplying top and bottom by the conjugate: $a-b\sqrt{5}$.
\end{hint*}

\begin{hint*}[Theorem~\ref{thm.BasicFieldProps}]
For the first part, notice that $x\cdot0 = x(0+0)$. For the last part, remember that the definition of a field ensures that $F$ has at least two elements, so there is some $a\in F$ with $a\neq 0$. Now, what happens if $0=1$?
\end{hint*}

\begin{hint*}[Theorem~\ref{thm.ZpField}]
The crux is to show that every nonzero element has a multiplicative inverse when $n$ is prime. Let $a\in (\mathbb{Z}_n)^*$. You need to find some integer $b$ such that $ab=1$ modulo $n$. Now, since $a\in (\mathbb{Z}_n)^*$ and  $n$ is prime, $\gcd(a,n) = 1$. By B\'ezout's Lemma, there exist $k,l\in \mathbb{Z}$ such that $1 = ka+ln$. What happens when you consider the equation $1 = ka+lp$ modulo $n$?
\end{hint*}

\begin{hint*}[Problem~\ref{prob.SubfieldRoot2PlusI}]
If $T_3$ is a subfield, then, in particular, it is closed under multiplication, so it must be that $\alpha^2\in T_3$. That means that $\alpha^2 = a+b\alpha$ for some $a,b\in \mathbb{Q}$. What does this imply?
\end{hint*}

\begin{hint*}[Problem~\ref{prob.QAdjoinI}]
Try following the approach in Example~\ref{exam.GenerateField}. First show  $\{a+bi\mid a,b\in \mathbb{Q}\} \subseteq \mathbb{Q}(i)$ by showing that every subfield that contains $\mathbb{Q}$ and $i$ must also contain $\{a+bi\mid a,b\in \mathbb{Q}\}$. To show the reverse containment, use the fact that $\{a+bi\mid a,b\in \mathbb{Q}\}$ is a subfield, by a previous problem.
\end{hint*}

\begin{hint*}[Problem~\ref{prob.QAdjoinRoot2PlusI}]
Remember, in Problem~\ref{prob.SubfieldRoot2PlusI}\ref{prob.SubfieldRoot2PlusI.QAdjoinRoot2PlusI}, we saw that $\{a+b\alpha\mid a,b\in \mathbb{Q}\}$ is \emph{not} a subfield of $\mathbb{C}$.
\end{hint*}

\begin{hint*}[Problem~\ref{prob.QAdjoinRoot27Root7}]
Use the previous theorem. To show  $\mathbb{Q}\left(3-\sqrt{2},5+i\right) \subseteq \mathbb{Q}\left(\sqrt{2},i\right)$, you need to show that $\mathbb{Q}\subset \mathbb{Q}\left(\sqrt{2},i\right)$ and that $3-\sqrt{2},5+i\in \mathbb{Q}\left(\sqrt{2},i\right)$. Then show the reverse containment in a similar way.
\end{hint*}

\begin{hint*}[Theorem~\ref{thm.SolvableByRadicalsNontrivialRootsOf1}]
Note that $x^n - 1 = (x-1)(x^{n-1} + x^{n-2} + \cdots + x^2 + x + 1)$. Now use Theorem~\ref{thm.nthRoots1}; note that $x^{n-1} + x^{n-2} + \cdots + x^2 + x + 1$ should only have  $n-1$ roots.
\end{hint*}

\begin{hint*}[Problem~\ref{prob.SolvableByRadicalsHard}]
First find the roots of $z^2 - 3z - 1$. Then, for each of those roots, use Theorem~\ref{thm.nthRoots} to solve for $z$. You should have 6 different roots in the end.
\end{hint*}

\begin{hint*}[Theorem~\ref{thm.UnitIsNotZeroDivisor}]
Try a proof by contradiction. Assume that $u$ is a unit and that $u$ is a zero divisor. Now, what does the definition of being a zero divisor tell you about $u$?
\end{hint*}

\begin{hint*}[Theorem~\ref{thm.DegreePolySum}]
To get started, let $n = \deg p(x)$ and $m = \deg q(x)$, and then write $p(x) = a_0 + a_1x + \cdots + a_nx^n$ with $a_n\neq 0$ and $q(x) = b_0 + b_1x + \cdots + b_mx^m$ with $b_m\neq 0$. You want to understand the degree of $p(x) + q(x)$, so you need to determine the largest power of $x$ in the sum $p(x) + q(x)$.
\end{hint*}

\begin{hint*}[Theorem~\ref{thm.DegreePolyProduct}]
As with the previous theorem, let $n = \deg p(x)$ and $m = \deg q(x)$, and then write $p(x) = a_0 + a_1x + \cdots + a_nx^n$ with $a_n\neq 0$ and $q(x) = b_0 + b_1x + \cdots + b_mx^m$ with $b_m\neq 0$. You need to determine the largest power of $x$ in the product $p(x)q(x)$. What do you think is the largest power of $x$ in the product $p(x)q(x)$? What is its coefficient, and how do you know it's not zero?
\end{hint*}

\begin{hint*}[Corollary~\ref{cor.PolysOverIntegralDomains}]
There are several things to verify to ensure that $D[x]$ is an integral domain, but we've talked about most of them already. The main thing that remains is to prove that $D[x]$ has no zero divisors---try a proof by contradiction. This is a corollary of Theorem~\ref{thm.DegreePolyProduct}, which means that it should be ``not too hard'' to prove using Theorem~\ref{thm.DegreePolyProduct}. 
\end{hint*}

\begin{hint*}[Theorem~\ref{thm.DivisionAlgorithm}]
One approach is to polish up and fill in the gaps of the outline presented in the notes right before the statement of Theorem~\ref{thm.DivisionAlgorithm}. A related, but slightly different, approach is to try using induction on the degree of $a(x)$.
\end{hint*}

\begin{hint*}[Theorem~\ref{thm.RootImpliesLinearFactorOfPoly}]
Try  using the division algorithm to write $a(x) = (x-c)q(x) + r(x)$ for some $q(x),r(x)\in F[x]$ with $\deg r(x) < \deg (x-c)$ or $r(x) = 0$. Now show that $r(x)$ must be the zero polynomial.
\end{hint*}

\begin{hint*}[Lemma~\ref{lem.GCDUnique}]
First, explain why $d_1(x)$ must divide $d_2(x)$ and why $d_2(x)$ must divide $d_1(x)$. Now return to the definition of ``to divide'' and see what you can write down.
\end{hint*}

\begin{hint*}[Theorem~\ref{thm.HalfOfGCDProof}]
Follow the definitions. Since $c(x)\in I$, it can be written a particular way. Then write down what it means for $h(x)$ to divide both $a(x)$ and $b(x)$. Combine.
\end{hint*}

\begin{hint*}[Theorem~\ref{thm.UnitsFAdjoinx}]
For the forward direction, start with the definition of a unit and apply the degree function. For the reverse direction, what does $\deg p(x) = 0$ imply about $p(x)$? Can you explicitly write down the a multiplicative inverse for $p(x)$?
\end{hint*}

\begin{hint*}[Theorem~\ref{thm.ReducibilityTestDegree2or3}]
Consider using Theorem~\ref{thm.LinearFactorOfPolyImpliesRoot}.
\end{hint*}

\begin{hint*}[Theorem~\ref{thm.FactorIrreducibles}]
Consider using using strong induction on the degree of the polynomial. Let $\varphi(n)$ be the statement ``every polynomial in $F[x]$ of degree $n$ can be written as a product of polynomials that are irreducible in $F[x]$.'' 

For the base case, you want to show that $\varphi(1)$ is true. Assume that $p(x)\in F[x]$ has degree $1$. Then what? 

Next, assume that $\varphi(k)$ is true for all $1\le k \le n$. We need to show that $\varphi(n+1)$ is true. Assume that $p(x)\in F[x]$ has degree $n+1$. There are two cases to consider: $p(x)$ is irreducible or $p(x)$ is reducible. Keep going\ldots
\end{hint*}









\end{document}