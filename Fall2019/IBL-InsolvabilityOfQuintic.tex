\documentclass[12pt,oneside]{book}

\usepackage[scale=2]{ccicons}
\usepackage{enumitem}
\usepackage{multicol}
\usepackage[labelsep=period]{caption}
\usepackage[labelformat=simple,labelfont={}]{subcaption}
\usepackage{makecell}
\usepackage{tabu}
\usepackage[table]{xcolor}
\usepackage{tikz}
\usetikzlibrary{arrows,automata,positioning}
\usetikzlibrary{decorations.pathreplacing}
\usepackage{rotating}
\usepackage[notextcomp]{kpfonts} 
\usepackage{graphicx}
\usepackage{eurosym}
\usepackage{amsfonts}
\usepackage{amsmath}
\usepackage{amssymb}
\usepackage{stmaryrd}
\usepackage{wasysym}
\usepackage{amsthm}
\usepackage[margin=1in]{geometry}
\usepackage[hang,flushmargin,symbol*]{footmisc}
\usepackage{color}
\definecolor{darkblue}{rgb}{0, 0, .6}
\definecolor{grey}{rgb}{.7, .7, .7}
\usepackage[breaklinks]{hyperref}
\hypersetup{
	colorlinks=true,
	linkcolor=darkblue,
	anchorcolor=darkblue,
	citecolor=darkblue,
	pagecolor=darkblue,
	urlcolor=darkblue,
	pdftitle={},
	pdfauthor={}
}

\usepackage{fancyhdr}
\pagestyle{fancy}
\lhead{\leftmark}
\chead{}
\rhead{}
\lfoot{}
\cfoot{\thepage}
\rfoot{}

\theoremstyle{definition}
\newtheorem{theorem}{Theorem}[chapter]
\newtheorem{acknowledgement}[theorem]{Acknowledgement}
\newtheorem{algorithm}[theorem]{Algorithm}
\newtheorem{axiom}[theorem]{Axiom}
\newtheorem{case}[theorem]{Case}
\newtheorem{claim}[theorem]{Claim}
\newtheorem{conclusion}[theorem]{Conclusion}
\newtheorem{condition}[theorem]{Condition}
\newtheorem{conjecture}[theorem]{Conjecture}
\newtheorem{corollary}[theorem]{Corollary}
\newtheorem{criterion}[theorem]{Criterion}
\newtheorem{definition}[theorem]{Definition}
\newtheorem{example}[theorem]{Example}
\newtheorem{exercise}[theorem]{Exercise}
\newtheorem{fact}[theorem]{Fact}
\newtheorem{journal}[theorem]{Journal}
\newtheorem{lemma}[theorem]{Lemma}
\newtheorem{notation}[theorem]{Notation}
\newtheorem{problem}[theorem]{Problem}
\newtheorem{proposition}[theorem]{Proposition}
\newtheorem{question}[theorem]{Question}
\newtheorem{remark}[theorem]{Remark}
\newtheorem{solution}[theorem]{Solution}
\newtheorem{summary}[theorem]{Summary}
\newtheorem{skeleton}[theorem]{Skeleton Proof}
\newtheorem{activity}[theorem]{Activity}
\newtheorem{intuitivedef}[theorem]{Intuitive Definition}

\newtheorem*{hint*}{Hint}
\newtheorem*{mainquestion}{Main Question}
\newtheorem*{maintheorem}{Main Theorem}
\newtheorem*{maincorollary}{Main Corollary}

\newsavebox{\savepar}
\newenvironment{textbox}{\noindent\begin{lrbox}{\savepar}\begin{minipage}[c]{.98\textwidth}}{\end{minipage}\end{lrbox}\fcolorbox{black}{white}{\usebox{\savepar}}}

\DeclareMathOperator{\Arg}{Arg}

\renewcommand\thesubfigure{(\alph{subfigure})}

\setenumerate[1]{label=\rm{(\arabic*)}}

\begin{document}

\title{Insolvability of the Quintic\\ {\small \normalfont An unofficial sequel to \href{https://github.com/dcernst/IBL-AbstractAlgebra}{An Inquiry-Based Approach to Abstract Algebra} by \href{http://danaernst.com}{Dana C. Ernst}}}
\author{Joshua Wiscons\\
California State University, Sacramento}
\date{Spring 2019}

\maketitle

\noindent\copyright{ \the\year\ Joshua Wiscons.  Some Rights Reserved.\\

\bigskip

\noindent The most up-to-date version of these notes on can be found on GitHub:
\begin{center}
\url{https://github.com/jwiscons/IBL-InsolvabilityOfQuintic}
\end{center}

\bigskip

\noindent This work is licensed under the Creative Commons Attribution-Share Alike 4.0 United States License.  You may copy, distribute, display, and perform this copyrighted work, but only if you give credit to Joshua Wiscons, and all derivative works based upon it must be published under the Creative Commons Attribution-Share Alike 4.0 International License. Please attribute this work to Joshua Wiscons, Mathematics Faculty at California State University, Sacramento, \url{joshua.wiscons@csus.edu}. To view a copy of this license, visit
\begin{center}
\url{https://creativecommons.org/licenses/by-sa/4.0/}
\end{center}
or send a letter to Creative Commons, 171 Second Street, Suite 300, San Francisco, California, 94105, USA.}

\medskip

\begin{center}
\ccbysa
\end{center}

\noindent This work is designed to extend \href{https://github.com/dcernst/IBL-AbstractAlgebra}{An Inquiry-Based Approach to Abstract Algebra} by \href{http://danaernst.com}{Dana C. Ernst}. The presentation of the material is heavily influenced by the book \href{https://www.amazon.com/Abstract-Algebra-Introduction-Robert-Redfield/dp/020143721X}{Abstract Algebra: A Concrete Introduction} by \href{https://www.hamilton.edu/academics/our-faculty/directory/faculty-detail/robert-redfield}{Robert H. Redfield}. Many thanks to both Dana and Bob! I am also indebted to every student in my Modern Algebra 2 class from Fall 2019 at California State University, Sacramento for taking part, willingly or otherwise, in this experiment---thanks to you all! 

\tableofcontents

% !TEX root = IBL-InsolvabilityOfQuintic.tex
\chapter{Introduction}

This course is a story. A repackaging of a famous story, spanning more-or-less 4000 years, about solving polynomial equations. I hope you like it. I also hope enjoy the beautiful sights along the way, many of which were a long time in the making and most of which are still being heavily researched to this day.

% % % % % % % % % % % % % % % % % % % % % % % % % % % % % % % % % % % % % % % % % % % %
% % % % % % % % % % % % % % % % % % % % % % % % % % % % % % % % % % % % % % % % % % % %
% SECTION
% % % % % % % % % % % % % % % % % % % % % % % % % % % % % % % % % % % % % % % % % % % %
% % % % % % % % % % % % % % % % % % % % % % % % % % % % % % % % % % % % % % % % % % % %
\section{Prerequisites}
A first course in abstract algebra, focusing on the basics of group theory, together with exposure to foundational topics like primes and divisibility, functions and relations, differential calculus, and linear algebra form the core prerequisites. Comfort with basic proof techniques, including induction, is also more-or-less required. The following two free and open-source books  by \href{https://danaernst.com}{Dana C.~Ernst} will serve you well if you need to review.
\begin{itemize}
\item \href{https://github.com/dcernst/IBL-AbstractAlgebra/blob/master/Fall2021/IBL-AbstractAlgebra.pdf}{An Inquiry-Based Approach to Abstract Algebra}
\item \href{https://github.com/dcernst/IBL-IntroToProof/blob/master/Fall2021/IntroToProof.pdf}{An Introduction to Proof via Inquiry-Based Learning}
\end{itemize}

% % % % % % % % % % % % % % % % % % % % % % % % % % % % % % % % % % % % % % % % % % % %
% % % % % % % % % % % % % % % % % % % % % % % % % % % % % % % % % % % % % % % % % % % %
% SECTION
% % % % % % % % % % % % % % % % % % % % % % % % % % % % % % % % % % % % % % % % % % % %
% % % % % % % % % % % % % % % % % % % % % % % % % % % % % % % % % % % % % % % % % % % %
\section{An Inquiry-Based Approach}
This section of the introduction as well as those that follow are (only slightly) modified from the introduction of  \href{https://github.com/dcernst/IBL-AbstractAlgebra/blob/master/Spring2018/IBL-AbstractAlgebra.pdf}{An Inquiry-Based Approach to Abstract Algebra}. The use of ``I'' below, does indeed refer to me, but mainly just because I also believe the words that \href{https://danaernst.com}{Dana} originally wrote.

In many courses, math or otherwise, you sit and listen to a lecture. These lectures may be polished and well-delivered. I love lecturing, and I do believe there is value in it. But, I also believe that in reality most students do not learn by simply listening. You must be active in the learning process. Likely, each of you have said to yourselves, ``Hmmm, I understood this concept when the professor was going over it, but now that I am alone, I am lost." To promote a more active participation in your learning, we will incorporate ideas from an educational philosophy called inquiry-based learning (IBL)\footnote{For more about IBL, check out \href{https://danaernst.com}{Dana's} blog post, \href{http://maamathedmatters.blogspot.com/2013/05/what-heck-is-ibl.html}{What the Heck is IBL?}}.

Loosely speaking, IBL is a student-centered method of teaching mathematics that engages students in sense-making activities.  Students are given tasks requiring them to solve problems, conjecture, experiment, explore, create, communicate.  Rather than showing just facts or a clear, smooth path to a solution, the instructor guides and mentors students via well-crafted problems through an adventure in mathematical discovery.  Effective IBL courses encourage deep engagement in rich mathematical activities and provide opportunities to collaborate with peers (either through class presentations or group-oriented work). I believe that there are two essential elements to IBL: students should as much as possible be responsible for
\begin{enumerate}
\item guiding the acquisition of knowledge;
\item validating the ideas presented (so students should not be looking to the instructor as the sole authority).
\end{enumerate}


Much of this course will be devoted to students discussing and proving theorems at the board because I believe that the best way to learn mathematics is by doing mathematics. Someone cannot master a musical instrument or a martial art by simply watching, and in a similar fashion, you cannot master mathematics by simply watching; you must do mathematics! In this class, students will regularly 
\begin{itemize}
\item read and interact with course notes on their own and with classmates;
\item write up their proofs to assigned problems;
\item discuss their proofs on the board to the rest of the class;
\item participate in discussions centered around a student's presented proof;
\item work to respond in flexible, thoughtful, and creative ways to problems that may seem unfamiliar on first glance.
\end{itemize}

It is important to understand that proving theorems is difficult and takes time. You should not expect to complete a proof in 10 minutes. Sometimes, you might have to stare at the statement for an hour before even understanding how to get started. However, there do exist some hints, collected in Appendix~\ref{chapter:Hints}. If you use the hints, please keep in mind that (1) your own learning will significantly benefit from cognitive struggles (independently and with your peers), so don't turn to the hints too early; and (2) the hints are really just some possible ways to get started. A hint might very well not be the way that makes the most sense to you, so I encourage you to follow you own path. 


Lastly, it is highly important to respect learning and to respect other people's ideas.  Whether you disagree or agree, please praise and encourage your fellow classmates.  Use ideas from others as a starting point rather than something to be judgmental about.  Judgement is not the same as being judgmental.  Helpfulness, encouragement, and compassion are highly valued.

% % % % % % % % % % % % % % % % % % % % % % % % % % % % % % % % % % % % % % % % % % % %
% % % % % % % % % % % % % % % % % % % % % % % % % % % % % % % % % % % % % % % % % % % %
% SECTION
% % % % % % % % % % % % % % % % % % % % % % % % % % % % % % % % % % % % % % % % % % % %
% % % % % % % % % % % % % % % % % % % % % % % % % % % % % % % % % % % % % % % % % % % %
\section{Structure of the Notes}
As you read the notes, you will be required to digest the material in a meaningful way.  It is your responsibility to read and understand new definitions and their related concepts.  However, you will be supported in this sometimes difficult endeavor. In addition, you will be asked to complete problems aimed at solidifying your understanding of the material.  Most importantly, you will be asked to make conjectures, produce counterexamples, and prove theorems.

The items labeled as \textbf{Definition}, \textbf{Example}, or \textbf{Fact} are meant to be read and digested.  However, the items labeled as \textbf{Problem}, \textbf{Lemma}, \textbf{Theorem}, and \textbf{Corollary} require action on your part.  Items labeled as \textbf{Problem} are sort of a mixed bag. Some Problems are computational in nature and aimed at improving your understanding of a particular concept while others ask you to provide a counterexample for a statement if it is false or to provide a proof if the statement is true. Items with the \textbf{Lemma}, \textbf{Theorem}, and \textbf{Corollary} designation are mathematical facts, and the intention is for you to produce a valid proof of the given statement.  \textbf{Lemma's} are usually stepping stones to the next theorem, though they are often  interesting in their own right. \textbf{Corollaries}  are typically statements that follow quickly from a previous theorem. In general, you should expect corollaries to have very short proofs.  However, that doesn't mean that you can't produce a more lengthy yet valid proof of a corollary. 

It is important to point out that there are very few examples in the notes.  This is intentional.  One of the goals of the items labeled as \textbf{Problem} is for \emph{you} to produce the examples.
 

\chapter{Solving polynomial equations and the main question}
\label{chapter:PolyEquations}
\thispagestyle{empty}

The story begins. Can you count all of the times in your career that you've had to find the zeros of a quadratic polynomial? What about a cubic polynomial? A quartic? Likely your answers are decreasing rapidly, and it's also likely that you have only solved cubic and quartic equations in very special situations. Why? Is it just that cubic and quartic equations are difficult to solve or could it be that some are impossible to ``solve.'' 

People have been investigating how to solve polynomial equations for about 4000 years. Let's get started.

\begin{problem}\label{prob.IntroFindZeros}
Determine the roots (i.e.~zeros) of each of the following. Try to use tools you've accumulated over the years, but you may well need a computer program (e.g. \href{https://www.wolframalpha.com}{WolframAlpha}) for some of them. For each problem, make a note about \emph{how} you found the roots. Try for exact answers, but you can approximate if needed.
\begin{multicols}{2}
\begin{enumerate}
\item $p(x) = x^2 - 5x + 6$
\item $q(x) = (x-3)^2 - 2$
\item $r(x) = x^2 + x + 1$
\item $s(x) = x^3-3x-2$ %rational root theorem
\item $f(x) = x^3 - 5$
\item $g(x) = x^4 - 1$
\item $a(x) = x^5 - 1$
\item $b(x) = x^5 +5 x^4-5$
\end{enumerate}
\end{multicols}
\end{problem}

\begin{problem}\label{prob.IntroFindZerosWhere}
For each part of Problem~\ref{prob.IntroFindZeros}, write down the ``smallest'' number system needed to express the roots of the given polynomial. Possible answers might be $\mathbb{Z}$ (integers), $\mathbb{Q}$ (rational numbers), $\mathbb{Z}$ together with $\sqrt{3}$, $\mathbb{Q}$ together with $\sqrt{-1}$ and $\sqrt[3]{5}$, etc.
\end{problem}

% % % % % % % % % % % % % % % % % % % % % % % % % % % % % % % % % % % % % % % % % % % %
% % % % % % % % % % % % % % % % % % % % % % % % % % % % % % % % % % % % % % % % % % % %
% SECTION
% % % % % % % % % % % % % % % % % % % % % % % % % % % % % % % % % % % % % % % % % % % %
% % % % % % % % % % % % % % % % % % % % % % % % % % % % % % % % % % % % % % % % % % % %
\begin{section}{Solving polynomial equations with formulas}
Though you may have solved the first three parts of Problem~\ref{prob.IntroFindZeros} different ways, there was one tool that would have solved them all: the quadratic formula. It will be valuable to (re)discover why it's true. 

First, let's slightly simplify things. Notice that $\alpha$ is a root of $ax^2 + bx + c$ if and only if $\alpha$ is a root of $x^2 + \frac{b}{a}x + \frac{c}{a}$ (assuming $a\neq 0$). This means that an arbitrary quadratic polynomial can always be converted to a quadratic polynomial whose leading coefficient is $1$ and in such a way that they have the same roots. Thus, if we have a formula that finds the roots of so-called \emph{monic} quadratic polynomials, we can actually use it to find the roots of \emph{all} quadratic polynomials.

\begin{definition}
A polynomial whose leading coefficient is $1$ is called a $\textbf{monic}$ polynomial.
\end{definition}

Now, let's (re)derive the quadratic formula for monic polynomials. Remember, we are \emph{(re)deriving} it, so \emph{please don't use the quadratic formula in your proof of the next theorem.}

\begin{theorem}\label{thm.MonicQuadratic}
The roots of $p(x) = x^2 + bx + c$ are \[\frac{-b\pm\sqrt{b^2 - 4c}}{2}.\]
\end{theorem}

\begin{problem}
Describe a number system, as small as possible, that is capable of expressing the roots of \emph{any} quadratic polynomial whose \emph{coefficients are all rational numbers}.
\end{problem}

Well, that takes care of quadratic polynomials. What about finding the roots of cubic polynomials? Well, it turns out that there is indeed a cubic formula, though it's decidedly more complicated than the quadratic formula. 

A method for deriving a cubic formula, due to \href{https://en.wikipedia.org/wiki/Scipione_del_Ferro}{Scipione del Ferro} and \href{https://en.wikipedia.org/wiki/Niccol�_Fontana_Tartaglia}{Tartaglia}, was published in a book by \href{https://en.wikipedia.org/wiki/Gerolamo_Cardano}{Cardano} in 1545. The starting point is to take a general cubic polynomial and first convert it to a monic polynomial (as we did above) and then convert it to a cubic of the form $x^3 + px +q$, always with the same roots as the original. Then, with work, one arrives at the following formula. 

\begin{fact}\label{fact:Cardano}
The roots of $p(x) = x^3 + px + q$ are \[\alpha + \beta, \left(-\frac{1}{2} + \frac{\sqrt{-3}}{2}\right)\alpha + \left(-\frac{1}{2} - \frac{\sqrt{-3}}{2}\right)\beta, \left(-\frac{1}{2} - \frac{\sqrt{-3}}{2}\right)\alpha + \left(-\frac{1}{2} + \frac{\sqrt{-3}}{2}\right)\beta\]
where $\alpha = \sqrt[3]{-\frac{q}{2} + \sqrt{\frac{q^2}{4}+\frac{p^3}{27}}}$ and $\beta = \sqrt[3]{-\frac{q}{2} - \sqrt{\frac{q^2}{4}+\frac{p^3}{27}}}$
\end{fact}

\begin{problem}
Describe a number system, as small as possible, that is capable of expressing the roots of \emph{any} cubic polynomial of the form $x^3 + px + q$ where $p,q\in \mathbb{Q}$.
\end{problem}

\begin{problem}
Use Fact~\ref{fact:Cardano} to write out the roots of $p(x) = x^3-2x-4$. Are any of the roots integers? Check your answer with \href{https://www.wolframalpha.com}{WolframAlpha}. Does this expose any issues with using the formula?
\end{problem}

For more details on solving cubic equations, you can use start with the \href{https://en.wikipedia.org/wiki/Cubic_function#Derivation_of_the_roots}{Wikipedia page about cubic functions}. And now\ldots quartic functions? Perhaps you have a guess.

\begin{problem}
Use the internet and/or library to determine if there is a quartic formula, i.e.~a formula to find the roots of fourth-degree polynomials. If there is a quartic formula, who are some people that discovered methods to derive it and when did they discover their method?
\end{problem}
\end{section}

% % % % % % % % % % % % % % % % % % % % % % % % % % % % % % % % % % % % % % % % % % % %
% % % % % % % % % % % % % % % % % % % % % % % % % % % % % % % % % % % % % % % % % % % %
% SECTION
% % % % % % % % % % % % % % % % % % % % % % % % % % % % % % % % % % % % % % % % % % % %
% % % % % % % % % % % % % % % % % % % % % % % % % % % % % % % % % % % % % % % % % % % %
\begin{section}{The main question(s)}
We now turn our attention to finding the roots of polynomials of degree five and higher. Well, what do you think: is there a quintic formula? Actually, there are two questions here: (1) what do we mean by ``formula'' (e.g.~what symbols/functions can we use), and (2) whatever we mean by formula, is there one that works for \emph{all} quintic polynomials?

You investigated two particular quintics in Problem~\ref{prob.IntroFindZeros}---what did you find? Did you find exact expressions for the roots? If so, \emph{how} were the roots expressed; that is, what symbols/functions were needed to write out the roots? 

Let's first try to tackle the ``what do we mean by formula'' question. Guided by the quadratic and cubic formulas, let's agree that we are looking for a formula that expresses the roots of an arbitrary quintic polynomial in terms of the coefficients of the polynomial using just the operations of addition, subtraction, multiplication, division, and the extraction of roots (square roots, cube roots,\ldots). This leads to the following intuitive definition, that we will work to sharpen later.

\begin{intuitivedef}
A polynomial is said to be \textbf{solvable by radicals} if every root of the polynomial can be expressed in terms of the coefficients of the polynomial using the operations of addition, subtraction, multiplication, division, and $\sqrt[n]{\phantom{x}}$ for any positive integer $n$.
\end{intuitivedef}

\begin{problem}
Find the roots of $p(x) = x^4 - 2x^2 -1$. Explain why $p(x)$ is solvable by radicals.
\end{problem}

\begin{problem}
Explain why every quadratic polynomial is solvable by radicals.
\end{problem}

Returning to quintic polynomials, our main question is as follows.

\begin{mainquestion} 
Does there exist a ``quintic formula'' that expresses the roots of an arbitrary quintic polynomial, in terms of the coefficients of the polynomial, using just rational numbers together with the operations of addition, subtraction, multiplication, division, and the extraction of roots?
\end{mainquestion}

Well, if the answer is yes, then it must be that every quintic polynomial is solvable by radicals. \textit{Spoiler Alert!}  Our goal (for the rest of the book!) is to prove the following theorem, in a rather elegant way. 

\begin{maintheorem}
Not every quintic polynomial is solvable by radicals.
\end{maintheorem}

\begin{maincorollary}
There is \emph{no} ``quintic formula'' that expresses the roots of an arbitrary quintic polynomial in terms of the coefficients of the polynomial using just the operations of addition, subtraction, multiplication, division, and the extraction of roots.
\end{maincorollary}

\begin{center}
Boom!
\end{center}

\end{section}









% !TEX root = IBL-InsolvabilityOfQuintic.tex
\chapter{Fields}
\label{chapter:Fields}
\thispagestyle{empty}


In Chapter~\ref{chapter:PolyEquations}, we explored problems about finding and expressing roots of polynomials, finally arriving at the goal of the course: proving that there are quintic polynomials that are \emph{not} solvable by radicals. This chapter serves two main purposes. First, as we look at roots of polynomials and how they can be expressed, it will be convenient to have a common world (i.e.~number system) in which they live. For us, this will be the complex numbers, denoted $\mathbb{C}$, which will be reviewed below. Our work with complex numbers will also supply the necessary language to properly talk about $n^{\text{th}}$-roots. Second, we are still in need of a proper definition of what it means for a polynomial to be ``solvable by radicals'', and this chapter lays the essential groundwork (to be completed in the next chapter). On the way to defining ``solvable by radicals'', we will be led to study the abstract the structure of $\mathbb{C}$ (and of $\mathbb{Q}$ and $\mathbb{R}$), arriving at the definition of a \emph{field}. 

% % % % % % % % % % % % % % % % % % % % % % % % % % % % % % % % % % % % % % % % % % % %
% % % % % % % % % % % % % % % % % % % % % % % % % % % % % % % % % % % % % % % % % % % %
% SECTION
% % % % % % % % % % % % % % % % % % % % % % % % % % % % % % % % % % % % % % % % % % % %
% % % % % % % % % % % % % % % % % % % % % % % % % % % % % % % % % % % % % % % % % % % %
\section{Complex Numbers}
As mentioned above, it will be convenient to work in a world that contains all of the roots of all of the polynomials that we will be studying. Considering the roots of polynomials such as $x^2 +1$, $x^2-2$, $x^2-3$, etc., we see that we need to include numbers like $\sqrt{-1}$, $\sqrt{2}$, $\sqrt{3}$, etc., so although there are smaller worlds one could choose, we will opt for the world containing both $\sqrt{-1}$ and $\mathbb{R}$, namely $\mathbb{C}$.

But before we proceed, note that $\sqrt{-1}$ is not really well defined. There are \emph{two} solutions to $x^2 +1$, so when we write $\sqrt{-1}$, we are all agreeing that we mean the same one. 

\begin{definition}
Let $i$ (or alternatively $\sqrt{-1}$) denote one particular solution to $x^2 +1 = 0$. 
\end{definition}

 Using $i$ and $\mathbb{R}$, we now build the complex numbers.

% % % % % % % % % % % % % % % % % % % % % % % % % % % % % % % % % % % % % % % % % % % %
% SUBSECTION
% % % % % % % % % % % % % % % % % % % % % % % % % % % % % % % % % % % % % % % % % % % %
\subsection{Definition and first principles}

\begin{definition}
The \textbf{complex numbers} is the set $\mathbb{C} := \{a + bi\mid a,b \in \mathbb{R}\}$. If $z=a+bi$, then $a$ is called the \textbf{real part} of $z$ and $b$ is called the \textbf{imaginary part} of $z$.
\end{definition}

Note that every complex number $z=a+bi$ is uniquely determined by two numbers: the real and imaginary parts $a$ and $b$. As such, we often graph complex numbers in the coordinate plane with the $x$-axis denoting the real part and the $y$-axis denoting the imaginary part. This will be called the \textbf{complex plane}.

\begin{example}
We graph $-2 + 2i$ and  $1 - 3i$ below.
\begin{center}
\begin{tikzpicture}[line width = .9,scale = .7]
\draw[<->] (-4,0) -- (4,0) node [below] {\small \textsc{Real}};
\draw[<->] (0,-4) -- (0,4) node [left] {\small \textsc{Imag.}};
\foreach \i in {1,2,3} {
\draw (-\i,-0.1) -- (-\i,0.1);
\draw (\i,-0.1) -- (\i,0.1);
\node[anchor = 90] at (-\i,0) {\small $-\i$};
\node[anchor = 90] at (\i,0) {\small $\i$};
\draw (-0.1,-\i) -- (0.1,-\i);
\draw (-0.1,\i) -- (0.1,\i);
\node[anchor = 0] at (0,-\i) {\small $-\i$};
\node[anchor = 0] at (0,\i) {\small $\i$};
}
\node (w) at (-2,2) {};
\fill (w) circle (.1);
\node[anchor = -10] at (w) {\small $-2 + 2i$};
\node (z) at (1,-3) {};
\fill (z) circle (.1);
\node[anchor = 170] at (z) {\small $1 - 3i$};
\end{tikzpicture}
\end{center}
\end{example}

We also define some operations on complex numbers.

\begin{definition}
We define the following operations on elements of $\mathbb{C}$. 
\begin{itemize}
\item \textbf{Addition:} $(a+bi) + (c+di) := (a+c) + (b+d)i$
\item \textbf{Multiplication:} $(a+bi) \cdot (c+di) := (ac-bd) + (ad+bc)i$
\item \textbf{Complex Conjugation:} $\overline{a+bi} := a-bi$
\end{itemize}
\end{definition}

Notice that in the definition of complex multiplication we are just using the normal distributive law (or FOIL if you like) together with the fact that $i^2 = -1$. Many of the familiar algebraic properties of $\mathbb{R}$ also hold for $\mathbb{C}$, which we will take as a fact.

\begin{fact}\label{fact.ComplexLaws} The following are true for $\mathbb{C}$.
\begin{itemize}
\item \textbf{Addition Laws:} Addition is associative and commutative. There is a unique additive identity, namely $0 = 0 + 0i$, and every number has a unique additive inverse, denoted $-(a+bi)$.
\item \textbf{Multiplication Laws:} Multiplication is associative and commutative. There is a unique multiplicative identity, namely $1 = 1 + 0i$, and every nonzero number has a unique multiplicative inverse, denoted $(a+bi)^{-1}$ or $\frac{1}{a+bi}$.
\item \textbf{Distributivity Laws:} For all $x,y,z \in \mathbb{C}$, $x(y+z) = xy+xz$ and $(y+z)x = yx+zx$.
\item \textbf{Conjugation Laws:} For all $x,y \in \mathbb{C}$, $\overline{x+y} = \overline{x} + \overline{y}$ and $\overline{x\cdot y} = \overline{x}\cdot\overline{y}$.
\end{itemize}
\end{fact}

\begin{problem}
Thinking of a complex number $z=a+bi$ as a point in the complex plane, describe \emph{geometrically} what happens when $(c+di)$ is added to $z$. Also, describe \emph{geometrically}  how to find $\overline{z}$ from $z$.
\end{problem}

When we plot points, there are different coordinate systems we could use. It turns out that  rectangular coordinates are good for adding complex numbers, but polar coordinates are better for multiplication. This lead to the following definition.

\begin{definition}
Let $z=a+bi$. 
\begin{enumerate}
\item The \textbf{modulus} of $z$, denoted $|z|$, is the \emph{radius} of the point $(a,b)$ when written in polar coordinates. Thus, $|z| = \sqrt{a^2 + b^2}$. 
\item The \textbf{argument} of $z$, denoted $\Arg(z)$, is the \emph{angle} in the interval $[0,2\pi)$ of the point $(a,b)$ when written in polar coordinates. Thus, $\cos(\Arg(z)) = \frac{a}{|z|}$ and $\sin(\Arg(z)) = \frac{b}{|z|}$ if $z\neq 0$. The argument of  $0$ is undefined.
\end{enumerate}
\end{definition}

\begin{example}
We have that $|-2 + 2i| = \sqrt{(-2)^2+2^2} = 2\sqrt{2}$ and $\Arg(-2 + 2i) = \frac{3\pi}{4}$. (But be careful, $\arctan\left(\frac{2}{-2}\right) = \frac{\pi}{4}$; you must pay attention to which quadrant the number is in.)
\begin{center}
\begin{tikzpicture}[line width = .9,scale = .7]
\small
\draw[<->] (-4,0) -- (4,0) node [below] { \textsc{Real}};
\draw[<->] (0,-3) -- (0,3) node [left] { \textsc{Imag.}};
\foreach \i in {1,2,3} {
\draw (-\i,-0.1) -- (-\i,0.1);
\draw (\i,-0.1) -- (\i,0.1);
\node[anchor = 90] at (-\i,0) { $-\i$};
\node[anchor = 90] at (\i,0) { $\i$};
}
\foreach \i in {1,2} {
\draw (-0.1,-\i) -- (0.1,-\i);
\draw (-0.1,\i) -- (0.1,\i);
\node[anchor = 0] at (0,-\i) { $-\i$};
\node[anchor = 0] at (0,\i) { $\i$};
}
\node (w) at (-2,2) {};
\fill (w) circle (.1);
\node[anchor = -10] at (w) { $-2 + 2i$};

\draw [dashed] (0,0) -- (w.center);
\draw [blue,domain=0:135,->] plot ({1.2*cos(\x)}, {1.2*sin(\x)});
\node[blue] at (1.25,1.25) {$\frac{3\pi}{4}$};

\draw [red,decorate,decoration={brace,amplitude=10pt},xshift=-4pt,yshift=0pt]
(0,0) -- (w.center);
\node[red] at (-2.2,0.6) {$2\sqrt{2}$}; 
\end{tikzpicture}
\end{center}
\end{example}

\begin{problem}\label{prob.ComplexCheckin}
For each of the following complex numbers,
\begin{itemize}
\item write it in the form $a+bi$ (if it is not already),
\item plot it in the complex plane,
\item find the modulus and argument (if not exact, then a decimal approximation is okay).
\end{itemize}
\begin{multicols}{2}
\begin{enumerate}
\item $u = -1-i$
\item $v= \frac{1}{1+i}$
\item $w = \frac{(2-i)(1+2i)}{2+3i}$
\item $z\in \mathbb{C}$ with $|z| = 3$ and $\Arg(z) = \frac{4\pi}{3}$
\end{enumerate}
\end{multicols}
\end{problem}

\begin{theorem}
Let $z\in \mathbb{C}$. If $z\neq 0$, then $z^{-1} = \displaystyle\frac{\overline{z}}{|z|^2}$.
\end{theorem}

The next theorem shows how to find an expression for a complex number given its modulus and argument.
 
\begin{theorem}\label{thm.PolarToRectangular}
Let $z\in \mathbb{C}$. Then $z = |z|\cos(\Arg(z)) + i|z|\sin(\Arg(z))$. Conversely, if $z = r\cos(\theta) + ir\sin(\theta)$, then 
$|z| = r$ and $\Arg(z) = \theta + 2\pi k$ for some $k\in \mathbb{Z}$.
\end{theorem}

We now derive some properties of multiplication. The first is quite useful and illustrates how multiplication is rather easy to deal with when numbers are in ``polar form''.

\begin{theorem}\label{thm.MultiplyComplex}
If $z_1 = r_1\cos\theta_1 + ir_1\sin\theta_1$ and $z_2=r_2\cos\theta_2 + ir_2\sin\theta_2$, then \[z_1z_2 = r_1r_2\cos(\theta_1+\theta_2) + ir_1r_2\sin(\theta_1+\theta_2).\]
\end{theorem}

\begin{corollary}
If $z_1,z_2\in \mathbb{C}$, then $|z_1z_2|=|z_1||z_2|$ and $\Arg(z_1z_2) = \Arg(z_1)+\Arg(z_2)+ 2\pi k$ for some $k\in \mathbb{Z}$.\end{corollary}

\begin{corollary}[De Moivre's formula]\label{cor.DeMoivre}
For each positive $n\in \mathbb{Z}$, \[\left(r\cos(\theta) + ir\sin(\theta)\right)^n = r^n\cos(n\theta) + ir^n\sin(n\theta).\]
\end{corollary}

% % % % % % % % % % % % % % % % % % % % % % % % % % % % % % % % % % % % % % % % % % % %
% SUBSECTION
% % % % % % % % % % % % % % % % % % % % % % % % % % % % % % % % % % % % % % % % % % % %
\subsection{Roots of unity}
We now arrive at an extremely important definition.

\begin{definition}
For each positive $n\in \mathbb{Z}$, define \[\zeta_n := \cos\left(\frac{2\pi}{n}\right) + i\sin\left(\frac{2\pi}{n}\right).\]
Thus, $\zeta_n$ (read as ``zeta n'') is the unique  number with magnitude $1$ and argument $\frac{2\pi}{n}$.
\end{definition}

\begin{problem}
Plot each of the following in the same complex plane: $\zeta_2$, $\zeta_3$, $\zeta_4$, $\zeta_5$.
\end{problem}

\begin{problem}
Plot each of the following in the same complex plane: $\zeta_6$, $(\zeta_6)^2$, $(\zeta_6)^3$, $(\zeta_6)^4$, $(\zeta_6)^5$, $(\zeta_6)^6$.
\end{problem}

\begin{problem}
Write $\overline{\zeta_8}$ as a power of $\zeta_8$. Conjecture a value for $\ell$ (in terms of $k$) such that $\overline{(\zeta_n)^k}=(\zeta_n)^\ell$; then prove that it works. Try starting with $k= 1$.
\end{problem}

We now turn our attention back to solving polynomial equations, focusing on those of the form $x^n - a$.

\begin{definition}\label{def.nthRoot}
Let $a\in \mathbb{C}$, and let $n\in \mathbb{Z}$ be positive. Then $z\in \mathbb{C}$ is called an \textbf{$n^\text{th}$ root of $a$} if $z^n = a$. In other words, the  $n^\text{th}$ roots of $a$ are the roots of the polynomial $x^n-a$. The  $n^\text{th}$ roots of $1$ are also called \textbf{$n^\text{th}$ roots of unity}.
\end{definition}

\begin{problem}\label{prob.nthRoots}
Let $w = -1 + i\sqrt{3}$. Write $w$ in polar form; then find a $4^\text{th}$ root of $w$.
\end{problem}

\begin{theorem}
For each non-negative $k\in \mathbb{Z}$, $(\zeta_n)^k$ is an $n^\text{th}$ root of $1$.
\end{theorem}

\begin{lemma}\label{lem.nthRoot1IsPowerOfZeta}
If $z$ is an $n^\text{th}$ root of $1$, then $z = (\zeta_n)^k$ for some non-negative $k\in \mathbb{Z}$.
\end{lemma}

\begin{lemma}\label{lem.ReducePowerOfZeta}
For each non-negative $k\in \mathbb{Z}$, $(\zeta_n)^k = (\zeta_n)^r$ for some $0\le r \le n-1$.
\end{lemma}

\begin{theorem}\label{thm.nthRoots1}
The set \[\{1, \zeta_n, (\zeta_n)^2, \ldots, (\zeta_n)^{n-1}\}\] is the set of \emph{all} $n^\text{th}$ roots of unity. Thus, there are $n$ distinct roots of $x^n-1$. 
\end{theorem}

\begin{lemma}
Let $a\in \mathbb{C}$ be nonzero, and let $b$ be any one particular $n^\text{th}$ root of $a$. Then $z$ is an $n^\text{th}$ root of $a$ if and only if $\frac{z}{b}$ is an $n^\text{th}$ root of $1$.
\end{lemma}

\begin{theorem}\label{thm.nthRoots}
Let $a\in \mathbb{C}$ be nonzero, and let $b$ be any one particular $n^\text{th}$ root of $a$. The set \[\{b, b\zeta_n, b(\zeta_n)^2, \ldots, b(\zeta_n)^{n-1}\}\] is the set of \emph{all} $n^\text{th}$ roots of $a$. Thus, there are $n$ distinct roots of $x^n-a$. 
\end{theorem}

\begin{problem}
Find \emph{all} roots of $x^3 - 8$, and also find \emph{all} roots of  $x^5 + 7$.
\end{problem}

% % % % % % % % % % % % % % % % % % % % % % % % % % % % % % % % % % % % % % % % % % % %
% SUBSECTION
% % % % % % % % % % % % % % % % % % % % % % % % % % % % % % % % % % % % % % % % % % % %
\subsection{Roots of polynomials over  $\mathbb{R}$ and  $\mathbb{C}$}
We conclude this section with a couple of general results about roots of polynomials.

\begin{theorem}\label{thm.RootsRealCoeff}
Suppose that $p(x) = a_nx^n + a_{n-1}x^{n-1} +\cdots+a_2x^2+a_1x+a_0$ with all $a_i\in \mathbb{R}$. If $z$ is a root of $p(x)$, then $\overline{z}$ is also a root of $p(x)$.
\end{theorem}

In words, the previous theorem says that if a polynomial has coefficients in $\mathbb{R}$, then the set of roots is ``closed under complex conjugation.'' We end with an extremely important theorem, which will be quite useful for us. However, since its proof is not our main goal (and since it requires sophisticated techniques), we will take it as fact.

\begin{fact}[Fundamental Theorem of Algebra]\label{fact.FTA}
If $p(x)$ is a non-constant polynomial with all coefficients in $\mathbb{C}$, then $p(x)$ has a root in $\mathbb{C}$.
\end{fact}

In fact, we will see that this implies that \emph{all} roots of such a $p(x)$ lie in $\mathbb{C}$, so in our of study polynomials (often with all coefficients even in $\mathbb{Q}$), $\mathbb{C}$ serves as a uniform world in which we can study the roots.

% % % % % % % % % % % % % % % % % % % % % % % % % % % % % % % % % % % % % % % % % % % %
% % % % % % % % % % % % % % % % % % % % % % % % % % % % % % % % % % % % % % % % % % % %
% SECTION
% % % % % % % % % % % % % % % % % % % % % % % % % % % % % % % % % % % % % % % % % % % %
% % % % % % % % % % % % % % % % % % % % % % % % % % % % % % % % % % % % % % % % % % % %
\section{An aside: the quaternions}
Our construction of the complex numbers creates a structure that contains the real numbers and possesses some nice properties not enjoyed by the real numbers, e.g.~every non-constant polynomial with complex coefficients has a complex root.  This raises the question: could we further extend the complex numbers to an even larger structure?

Concisely, we built the complex numbers as the set $\mathbb{C} = \mathbb{R} + \mathbb{R}i$ together with the operations of addition and multiplication, which were defined in a natural way from the key identity that $i^2 = -1$. Here, we briefly explore what happens if we build a larger structure in a similar way: $\mathbb{H} = \mathbb{C} + \mathbb{C}j$ where, again, $j^2 = -1$. 

Following this path, we formally arrive at $\mathbb{H} = \mathbb{C} + \mathbb{C}j= (\mathbb{R} + \mathbb{R}i) + (\mathbb{R} + \mathbb{R}i)j$, and any definition we give for multiplication of two elements of $\mathbb{H}$ must first define how to multiply $i$ and $j$ (or rather, what properties $ij$ should have). If we set $k=ij$, it turns out that a good route to follow is to decide that $k$ also has the property that it squares to 1, i.e.~$k^2 = -1$. There is another important choice one is ``forced'' to make, namely that $ji = -k$.

\begin{definition}
The \textbf{quaternions} are the elements of $\mathbb{H} := \{a + bi+ cj + dk\mid a,b,c,d \in \mathbb{R}\}$, where $i^2 = j^2 = k^2= -1$. We also define the following operations on elements of $\mathbb{H}$.
\begin{itemize}
\item \textbf{Addition:} $(a_1 + b_1i+ c_1j + d_1k) + (a_2 + b_2i+ c_2j + d_2k) := (a_1+a_2) + (b_1+b_2)i+ (c_1+c_2)j + (d_1+d_2)k$
\item \textbf{Multiplication:} use the usual distributive laws together with the identities:
\[\begin{array}{lll}
ij = k, &jk = i, &ki = j,\\
ji = -k, &kj = -i, &ik = -j.
\end{array}\]
\item \textbf{Conjugation:} $\overline{a + bi+ cj + dk} := a - bi -cj - dk$
\end{itemize}
\end{definition}

Indeed, $\mathbb{H}$ extends the complex numbers, and we have the following containments: $\mathbb{Q}\subset\mathbb{R}\subset\mathbb{C}\subset\mathbb{H}$. It turns out that $\mathbb{H}$ satisfies nearly all of the common algebraic properties of $\mathbb{R}$ and $\mathbb{C}$, with one notable exception, which is highlighted in bold below.

\begin{fact}\label{fact.QuaternionLaws} The following are true for $\mathbb{H}$.
\begin{itemize}
\item \textbf{Addition Laws:} Addition is associative and commutative. There is a unique additive identity, namely $0 = 0 + 0i+0j+0k$, and every number has a unique additive inverse.
\item \textbf{Multiplication Laws:} Multiplication is associative but \textbf{noncommutative}. There is a unique multiplicative identity, namely $1 = 1 + 0i+0j+0k$, and every nonzero number has a unique multiplicative inverse.
\item \textbf{Distributivity Laws:} For all $x,y,z \in \mathbb{H}$, $x(y+z) = xy+xz$ and $(y+z)x = yx+zx$.
\item \textbf{Conjugation Laws:} For all $x,y \in \mathbb{H}$, $\overline{x+y} = \overline{x} + \overline{y}$ and $\overline{x\cdot y} = \overline{x}\cdot\overline{y}$.
\end{itemize}
\end{fact}

We can also define the modulus of a quaternion, analogous to how we defined it for a complex number.

\begin{definition}
The \textbf{modulus} of $h=a + bi+ cj + dk$ , denoted $|h|$, is $|h| = \sqrt{a^2 + b^2+c^2+d^2}$. 
\end{definition}

\begin{problem}\label{prob.QuaternionCheckin}
Write each of the following quaternions in the form $a+bi+cj+dk$, and find its modulus.
\begin{multicols}{2}
\begin{enumerate}
\item $h = (2i+4k)(7-3j+k)$
\item $k^{-1}$
\item $w = (i+j)^{-1}$
\end{enumerate}
\end{multicols}
\end{problem}

\begin{theorem}
Let $h\in \mathbb{H}$. If $h\neq 0$, then $h^{-1} = \displaystyle\frac{\overline{h}}{|h|^2}$.
\end{theorem}

And as in the complex numbers, the modulus function is multiplicative---we will take this as fact.

\begin{fact}
If $h_1,h_2\in \mathbb{H}$, then $|h_1h_2| = |h_1||h_2|$.
\end{fact}

We conclude this section by looking at multiplication of quaternions a little closer. As we do, we return to the mathematical notion of a \emph{group}. Please feel free to look over old notes or other books to review the basics. As mentioned in the introduction, our main reference for groups will be  \href{https://github.com/dcernst/IBL-AbstractAlgebra/blob/master/Fall2021/IBL-AbstractAlgebra.pdf}{An Inquiry-Based Approach to Abstract Algebra}.

\begin{problem}
Let $G$ be the subset of $\mathbb{H}$ defined as $G:=\{\pm1,\pm i,\pm j,\pm k\}$. Show that $G$, together with the operation of quaternion multiplication, is a nonabelian group. If you have encountered this group before, what name (or symbol) did you know it by?
\end{problem}

\begin{problem}
Let $U$ be the subset of $\mathbb{H}$ consisting of all quaternions with modulus equal to $1$, i.e.~$U:=\{h\in \mathbb{H}\mid |h| = 1\}$. Show that $U$, together with the operation of quaternion multiplication, is an infinite, nonabelian group.
\end{problem}

It turns out that the group $U$ from the previous problem is isomorphic to the group $\operatorname{SU}(2)$ (one of the the so-called special unitary groups), which is quite important in theoretical physics. If you want to learn more, you can start on \href{https://en.wikipedia.org/wiki/Special_unitary_group#The_group_SU(2)}{Wikipedia}.

% % % % % % % % % % % % % % % % % % % % % % % % % % % % % % % % % % % % % % % % % % % %
% % % % % % % % % % % % % % % % % % % % % % % % % % % % % % % % % % % % % % % % % % % %
% SECTION
% % % % % % % % % % % % % % % % % % % % % % % % % % % % % % % % % % % % % % % % % % % %
% % % % % % % % % % % % % % % % % % % % % % % % % % % % % % % % % % % % % % % % % % % %
\section{Abstract fields}
Notice that $\mathbb{Q}$, $\mathbb{R}$, and $\mathbb{C}$ satisfy many common algebraic properties with respect to addition and multiplication. Of course, $\mathbb{H}$ does too, though it lacks commutativity of multiplication. When objects have common properties, it can be extremely valuable to abstract  those properties and study them once and for all (as opposed to trying to prove things about each individual structure). This is where we are headed, but first we highlight some related structures (again with algebraic properties similar to $\mathbb{Q}$, $\mathbb{R}$, and $\mathbb{C}$) that help to connect this work to our main goal of expressing roots of polynomials.

\begin{problem}
Let $p(x) = x^2+3x+1$. Find the roots of $p(x)$, and show that each root can be written in the form $a+b\sqrt{5}$ with $a,b\in \mathbb{Q}$.
\end{problem}

\begin{problem}\label{prob.QAdjoinRoot5Closure}
Let $S=\{a+b\sqrt{5}\mid a,b\in \mathbb{Q}\}$. 
\begin{enumerate}
\item Show that $S$ is closed under addition; that is, show that for all $x,y\in S$, $x+y\in S$.
\item Show that $S$ is closed under multiplication; that is, show that for all $x,y\in S$, $xy\in S$.
\item Use that $S\subset \mathbb{R}$ to explain why both addition and multiplication of elements of $S$ are associative and commutative and why multiplication distributes over addition.
\end{enumerate}
\end{problem}

\begin{problem}\label{prob.QAdjoinRoot5Inverse}
Let $S=\{a+b\sqrt{5}\mid a,b\in \mathbb{Q}\}$. Prove or disprove: if $x\in S$ and $x\neq 0$, then $x$ has a multiplicative inverse \emph{in $S$} (i.e.~there is a $y\in S$ such that $xy=1$).
\end{problem}

% % % % % % % % % % % % % % % % % % % % % % % % % % % % % % % % % % % % % % % % % % % %
% SUBSECTION
% % % % % % % % % % % % % % % % % % % % % % % % % % % % % % % % % % % % % % % % % % % %
\subsection{Definition}

We now abstract the common properties of $\mathbb{Q}$, $\mathbb{R}$, and $\mathbb{C}$ (and also $S$ from Problem~\ref{prob.QAdjoinRoot5Closure}), arriving at the definition of a field. 

\begin{definition}
A \textbf{field} is a structure $(F,+,\cdot)$ consisting of a set $F$, containing at least two elements, together with two binary operations $+$ and $\cdot$ (which we call \emph{addition} and \emph{multiplication}) such that for some elements $0,1\in F$ the following axioms hold.
\begin{itemize}
\item \textbf{Addition Axioms:} Addition is associative and commutative; the element $0$ is an additive identity; every  $x\in F$ has an additive inverse with respect to $0$, denoted $-x$.
\item \textbf{Multiplication Axioms:} Multiplication is associative and commutative; the element $1$ is a multiplicative identity;  every $x\in F\setminus\{0\}$ has a multiplicative inverse with respect to $1$, denoted $x^{-1}$.
\item \textbf{Distributivity Axioms:} For all $x,y,z \in F$, $x(y+z) = xy+xz$ and $(y+z)x = yx+zx$.
\end{itemize}
\end{definition}

Recall that ``$0$ is an additive identity'' means that ``for all $x\in F$, $0+x = x+0 = x$,'' and ``$x\in F$ has an additive inverse with respect to $0$'' means that ``there exists some $y\in F$ such that $x+y = y+x = 0$.'' The meanings of multiplicative identities and inverses are similar to those for addition. Also, recall that $F\setminus\{0\}$ denotes the set obtained by removing the element $0$ from $F$. We introduce some notation for this.

\begin{definition}
If $F$ is a field, then $F\setminus\{0\}$ is denoted by $F^*$, i.e.~$F^*$ is the set of nonzero elements of $F$.
\end{definition}

Using the language of groups, fields can be concisely defined as structures of the form $(F,+,\cdot)$ such that $(F,+)$ is an abelian group with identity $0$,  $(F^*,\cdot)$ is an abelian group with identity $1$, and multiplication distributes over addition.

Now, as with any new definition, we look for examples and basic properties.

% % % % % % % % % % % % % % % % % % % % % % % % % % % % % % % % % % % % % % % % % % % %
% SUBSECTION
% % % % % % % % % % % % % % % % % % % % % % % % % % % % % % % % % % % % % % % % % % % %
\subsection{Examples and non-examples}

It is not hard to verify that $\mathbb{Q}$, $\mathbb{R}$, $\mathbb{C}$, and $S$ from Problem~\ref{prob.QAdjoinRoot5Closure} are all fields (with their usual definitions of addition and multiplication). Let's search for more examples and non-examples.

\begin{problem}
Explain why $\mathbb{Z}$ is not a field. Do the same for $\mathbb{H}$.
\end{problem}

\begin{problem}\label{prob.FieldExampleTable}
Determine if each of the following is a field. If it is a field, identify an additive and multiplicative identity; if it is not a field, explain why not.
\begin{enumerate}
\item $(F,+,\cdot)$ where $F=\{a,b,c\}$ and $+$ and $\cdot$ are defined as follows:
\begin{center}
\begin{tabu}{c|[2pt]c|c|c}
$+$ & $a$ & $b$ & $c$ \\ \tabucline[2pt]{-}
$a$ & $b$ & $c$ & $a$ \\ \hline 
$b$ & $c$ & $a$ & $b$ \\ \hline 
$c$ & $a$ & $b$ & $c$
\end{tabu}
\hspace{.5in}
\begin{tabu}{c|[2pt]c|c|c}
$\cdot$ & $a$ & $b$ & $c$ \\ \tabucline[2pt]{-}
$a$ & $a$ & $b$ & $c$ \\ \hline 
$b$ & $b$ & $a$ & $c$ \\ \hline 
$c$ & $c$ & $c$ & $c$
\end{tabu}
\end{center}
\item $(F,+,\cdot)$ where $F=\{0,1,2,3\}$ and $+$ and $\cdot$ are defined as follows:
\begin{center}
\begin{tabu}{c|[2pt]c|c|c|c}
$+$ & $0$ & $1$ & $2$ & $3$ \\ \tabucline[2pt]{-}
$0$ & $0$ & $1$ & $2$ & $3$ \\ \hline 
$1$ & $1$ & $2$ & $3$ & $0$ \\ \hline 
$2$ & $2$ & $3$ & $0$ & $1$ \\ \hline
$3$ & $3$ & $0$ & $1$ & $2$
\end{tabu}
\hspace{.5in}
\begin{tabu}{c|[2pt]c|c|c|c}
$\cdot$ & $0$ & $1$ & $2$ & $3$ \\ \tabucline[2pt]{-}
$0$ & $0$ & $0$ & $0$ & $0$ \\ \hline 
$1$ & $0$ & $1$ & $2$ & $3$ \\ \hline 
$2$ & $0$ & $2$ & $0$ & $2$ \\ \hline
$3$ & $0$ & $3$ & $2$ & $1$
\end{tabu}
\end{center}

\item\label{prob.FieldExampleTable.F4} $(F,+,\cdot)$ where $F=\{0,1,2,3\}$ and $+$ and $\cdot$ are defined as follows:
\begin{center}
\begin{tabu}{c|[2pt]c|c|c|c}
$+$ & $0$ & $1$ & $2$ & $3$ \\ \tabucline[2pt]{-}
$0$ & $0$ & $1$ & $2$ & $3$ \\ \hline 
$1$ & $1$ & $0$ & $3$ & $2$ \\ \hline 
$2$ & $2$ & $3$ & $0$ & $1$ \\ \hline
$3$ & $3$ & $2$ & $1$ & $0$
\end{tabu}
\hspace{.5in}
\begin{tabu}{c|[2pt]c|c|c|c}
$\cdot$ & $0$ & $1$ & $2$ & $3$ \\ \tabucline[2pt]{-}
$0$ & $0$ & $0$ & $0$ & $0$ \\ \hline 
$1$ & $0$ & $1$ & $2$ & $3$ \\ \hline 
$2$ & $0$ & $2$ & $3$ & $1$ \\ \hline
$3$ & $0$ & $3$ & $1$ & $2$
\end{tabu}
\end{center}
\end{enumerate}
\end{problem}

\begin{problem}
Look back at Problem~\ref{prob.FieldExampleTable}. For those that are fields, determine which familiar group each of $(F,+)$ and $(F^*,\cdot)$ is isomorphic to.
\end{problem}

To find more examples of fields, Problem~\ref{prob.FieldExampleTable} hints at the fact we may want to look back to modular arithmetic. Following \href{https://github.com/dcernst/IBL-AbstractAlgebra}{An Inquiry-Based Approach to Abstract Algebra}, we define the structures $(\mathbb{Z}_n,+_n,\cdot_n)$ as follows.

\begin{definition}
Let $n$ be a positive integer. The structure $(\mathbb{Z}_n,+_n,\cdot_n)$ consists of the set $\mathbb{Z}_n := \{0,1,2,\ldots,n-1\}$ together with the operations $+_n$ and $\cdot_n$ defined as follows.
\begin{itemize}
\item \textbf{Addition:} $x +_n y$ is the least non-negative number congruent to $x + y$ modulo $n$.
\item \textbf{Multiplication:} $x \cdot_n y$ is the least non-negative number congruent to $x \cdot y$ modulo $n$.
\end{itemize}
We often refer to the entire structure $(\mathbb{Z}_n,+_n,\cdot_n)$ as simply $\mathbb{Z}_n$. Also, when the context is clear, we may write $+$ and $\cdot$ in place of $+_n$ and $\cdot_n$.
\end{definition}

So, in $\mathbb{Z}_5$, we write equations like $3+6 = 4$, since $3+6 = 9$ and $9$ is congruent to $4$ when working modulo $5$. If needed, we can highlight that we are working modulo $5$ by writing  $3+6 \equiv_5 4$. And with respect to multiplication in $\mathbb{Z}_5$, we have equations like $2\cdot3 = 1$, which implies that $3$ is a multiplicative inverse of $2$ (and vice versa) in $\mathbb{Z}_5$.

\begin{fact}\label{fact.IntegersModnLaws} The following are true for $\mathbb{Z}_n$.
\begin{itemize}
\item \textbf{Addition Laws:} Addition is associative and commutative. There is a unique additive identity, namely $0$; each $x\in\mathbb{Z}_n$ has a unique additive inverse, denoted $-x$.
\item \textbf{Multiplication Laws:} Multiplication is associative and commutative. There is a unique multiplicative identity, namely $1$.
\item \textbf{Distributivity Laws:} For all $x,y,z \in \mathbb{Z}_n$, $x(y+z) = xy+xz$ and $(y+z)x = yx+zx$.
\end{itemize}
\end{fact}

\begin{problem}
Show that $\mathbb{Z}_5$ is a field but $\mathbb{Z}_6$ is not.
\end{problem}

\begin{problem}\label{prob.ConjectureZn}
Make a conjecture as to when $\mathbb{Z}_n$ is a field. That is, try to fill in the blank: ``$\mathbb{Z}_n$ is a field provided \fillInBlank{something about $n$}.'' What evidence do you have?
\end{problem}

% % % % % % % % % % % % % % % % % % % % % % % % % % % % % % % % % % % % % % % % % % % %
% SUBSECTION
% % % % % % % % % % % % % % % % % % % % % % % % % % % % % % % % % % % % % % % % % % % %
\subsection{Basic properties}

Let's explore some basic properties of fields. We list some of these as facts since they follow directly from group theory, remembering that $(F,+)$ and $(F^*,\cdot)$ are both groups. 

From now on, when we write ``let $F$ be a field,'' we tacitly mean ``let $(F,+,\cdot)$ be a field.''

\begin{fact}\label{thm.BasicFieldPropsUniqueness}
Let $F$ be a field. 
\begin{enumerate}
\item The additive identity and the multiplicative identity are both unique.
\item Additive inverses and multiplicative inverses are unique.
\end{enumerate}
\end{fact}

\begin{theorem}\label{thm.BasicFieldProps}
Let $F$ be a field. 
\begin{enumerate}
\item For all $x\in F$, $x\cdot0 = 0$.
\item For all $x,y\in F$, $(-x)y = -(xy)$ and $x(-y) = -(xy)$.
\item For all $x\in F^*$, $-x\in F^*$ and $(-x)^{-1} = -(x^{-1})$.
\item\label{thm.BasicFieldProps.NoZeroDivisors} For all $x,y\in F$, if $xy = 0$, then $x=0$ or $y=0$.
\item The additive and multiplicative identities are different, i.e.~$0\neq 1$.
\end{enumerate}
\end{theorem}

% % % % % % % % % % % % % % % % % % % % % % % % % % % % % % % % % % % % % % % % % % % %
% SUBSECTION
% % % % % % % % % % % % % % % % % % % % % % % % % % % % % % % % % % % % % % % % % % % %
\subsection{Another example}
We now return to the conjecture you made in Problem~\ref{prob.ConjectureZn}. Combining the next theorem with Theorem~\ref{thm.BasicFieldProps}, we see that $\mathbb{Z}_n$ has no hope to be a field unless $n$ is prime.

\begin{theorem}
Let $n$ be a positive integer. If $n$ is not prime, then there exist $a,b\in (\mathbb{Z}_n)^*$ such that $ab=0$ in $\mathbb{Z}_n$.
\end{theorem}

And now we completely answer the question. As you explore the next theorem, you can use properties of modular arithmetic that you know from before. For example, you can take for granted that addition and multiplication are both associative and commutative. The crux is in showing that every nonzero element has a multiplicative inverse when $n$ is prime. There are many ways to approach this; one way uses \href{https://en.wikipedia.org/wiki/Bezout\%27s_identity}{B\'ezout's lemma} from basic number theory. Even if you don't use it now, it's a  useful fact to remember.

\begin{fact}[B\'ezout's lemma]\label{Fact.Bezout}
If $a,b\in \mathbb{Z}$, then there exist $k,l\in \mathbb{Z}$ such that $ka+lb = \gcd(a,b)$.
\end{fact}

\begin{theorem}\label{thm.ZpField}
Let $n$ be a positive integer. Then $\mathbb{Z}_n$ is a field if and only if $n$ is prime.
\end{theorem}

% % % % % % % % % % % % % % % % % % % % % % % % % % % % % % % % % % % % % % % % % % % %
% % % % % % % % % % % % % % % % % % % % % % % % % % % % % % % % % % % % % % % % % % % %
% SECTION
% % % % % % % % % % % % % % % % % % % % % % % % % % % % % % % % % % % % % % % % % % % %
% % % % % % % % % % % % % % % % % % % % % % % % % % % % % % % % % % % % % % % % % % % %
\section{Subfields and extension fields}

Just as with groups and subgroups, the notion of a subfield is extremely important. Analyzing the subfields of a field $F$ can often yield a better understanding of the whole field $F$, and vice versa. Also, this will allow us to generate  more examples of fields.

\begin{definition}
Let $(E,+,\cdot)$ be a field, and let $F$ be a subset of $E$. Then $F$ is a \textbf{subfield} of $E$ if $F$ is a field in its own right with respect to operations $+$ and $\cdot$ \emph{inherited from $E$}. When $F$ is a subfield of $E$, we call $E$  an \textbf{extension field} of $F$.
\end{definition}

When checking if a subset of a field is a subfield, it turns out that the subset will automatically satisfy many of the field axioms, leaving only a handful of things to verify.

\begin{theorem}
Let $E$ be a field, and let $F\subseteq E$. Then $F$ is a subfield of $E$ if and only if 
\begin{enumerate}
\item $F$ contains at least $2$ elements;
\item for all $x,y\in F$, $x+y\in F$ and  $xy\in F$;
\item for all $x\in F$, $-x\in F$; and 
\item for all $x\in F^*$, $x^{-1}\in F$.
\end{enumerate}
\end{theorem}

The second item in the above theorem is stating that $F$ is closed under the addition and multiplication inherited from $E$. The last two items could be read as $F$ being closed under additive and multiplicative inverses.

\begin{theorem}\label{thm.SubfieldContains01}
If $F$ is a subfield of $E$, then $F$ contains the additive and multiplicative identities of $E$ (namely $0$ and $1$).
\end{theorem}

It is not difficult to check that $\mathbb{Q}$ and $\mathbb{R}$ are both subfields of $\mathbb{C}$; $S$ from Problem~\ref{prob.QAdjoinRoot5Closure} is also a subfield of $\mathbb{C}$ (and of $\mathbb{R}$). Let's look for more that are similar to $S$.

\begin{problem}\label{prob.SubfieldRoot2PlusI}
Determine which of the following are subfields of $\mathbb{C}$.
\begin{enumerate}
\item\label{prob.SubfieldRoot2PlusI.QAdjoinI} $T_1=\{a+bi\mid a,b\in \mathbb{Q}\}$
\item $T_2=\{a+bi\mid a,b\in \mathbb{Z}\}$
\item\label{prob.SubfieldRoot2PlusI.QAdjoinRoot2PlusI} $T_3=\{a+b\alpha\mid a,b\in \mathbb{Q}\}$ where  $\alpha = \sqrt{2} + i$
\end{enumerate}
\end{problem}


% % % % % % % % % % % % % % % % % % % % % % % % % % % % % % % % % % % % % % % % % % % %
% SUBSECTION
% % % % % % % % % % % % % % % % % % % % % % % % % % % % % % % % % % % % % % % % % % % %
\subsection{Generating fields}
Paralleling the theory of groups, we now investigate how to generate subfields from subsets of elements. We first need a \emph{definition} of ``the subfield generated by a set of elements'';  it is essentially the same as for all algebraic structures: take the intersection of all subfields containing the subset.

\begin{theorem}\label{thm.IntersectFields}
If $F_1$ and $F_2$ are subfields of a field $E$, then $F_1\cap F_2$ is a subfield of $E$.
\end{theorem}

This can be generalized to intersections of an arbitrarily collection of subfields . 

\begin{theorem}\label{thm.IntersectCollectionFields}
If $\mathcal{C}$ is any collection of subfields of a field $E$, then the intersection of all subfields from $\mathcal{C}$ is again an subfield of $E$.
\end{theorem}


If $S$ is any subset of a field $E$, we can let $\mathcal{C}$ be the collection of all subfields containing $S$, and we now know that the intersection of subfields in $\mathcal{C}$ will yield \emph{subfield}, which then must be the ``smallest'' subfield containing $S$. This leads to the following definition. 


\begin{definition}\label{def.GenerateField}
Suppose $S$ is a subset of a field $E$. The \textbf{subfield of $E$ generated by $S$}, denoted $\langle S \rangle_{\textsc{field}}$, is defined to be the intersection of all subfields of $E$ that contain $S$.
\end{definition}

\begin{remark}
By the definition of $\langle S \rangle_{\textsc{field}}$, we know that for all subfields $F$ of a field $E$, if $S\subseteq F$, then $\langle S \rangle_{\textsc{field}} \subseteq F$.
\end{remark}

\begin{example}\label{exam.GenerateField}
Let's explore  $\langle 1 \rangle_{\textsc{field}}$ in the field $\mathbb{C}$. By definition,  $\langle 1 \rangle_{\textsc{field}}$ is the intersection of all subfields of $\mathbb{C}$ that contain $1$. 

Let $F$ be an arbitrary subfield of $\mathbb{C}$ containing $1$. By Theorem~\ref{thm.SubfieldContains01}, every subfield of $\mathbb{C}$ contains $0$ and $1$, so $F$ must  contain $0$ (in addition to $1$). Further, $F$ must contain $1+1$, $1+1+1$, etc., because $F$ is closed under addition. So, by induction,  $F$ contains the positive integers and $0$. Then, since $F$ is closed under additive inverses, $F$ also contains the additive inverse of each positive integer, so in total, we now see that $F$ contains $\mathbb{Z}$. Continuing on, $F$ is closed under multiplicative inverses, so $F$ also contains the multiplicative inverse of every nonzero integer. Thus, $\mathbb{Q} \subseteq F$. 

Since $F$ was an \emph{arbitrary} subfield of $\mathbb{C}$ containing $1$, everything we said above  is true for \emph{every} subfield of $\mathbb{C}$ containing $1$; thus it is also true for the intersection of them. Hence $\mathbb{Q} \subseteq \langle 1 \rangle_{\textsc{field}}$. Now we have $\{1\} \subset \mathbb{Q} \subseteq \langle 1 \rangle_{\textsc{field}}$, so as $\mathbb{Q}$ is a subfield and $\langle 1 \rangle_{\textsc{field}}$ is the \emph{smallest} subfield containing $1$, it must be that $\mathbb{Q} = \langle 1 \rangle_{\textsc{field}}$.
\end{example}


\begin{theorem}\label{thm.BaseFieldC}
If $S\subseteq \mathbb{C}$, then  $\mathbb{Q} \subseteq \langle S \rangle_{\textsc{field}}$.
\end{theorem}

\begin{problem}
The field defined in Problem~\ref{prob.FieldExampleTable}\ref{prob.FieldExampleTable.F4} is sometimes denoted $\mathbb{F}_4$. Determine $\langle 1 \rangle_{\textsc{field}}$ in the field $\mathbb{F}_4$.
\end{problem}

Most of the time, we will want to generate fields by adding some elements to an existing field, and we have special notation for this.

\begin{notation}
Let $F$ be a subfield of $E$, and let $r_1,r_2,\ldots,r_n \in E$. The subfield of $E$ generated by $F\cup \{r_1,r_2,\ldots,r_n\}$ is denoted $F(r_1,r_2,\ldots,r_n)$. In other words, \[F(r_1,r_2,\ldots,r_n) := \langle F\cup \{r_1,r_2,\ldots,r_n\} \rangle_{\textsc{field}}.\]
\end{notation}

We read $F(r_1,r_2,\ldots,r_n)$  as ``$F$ adjoin $r_1,r_2,\ldots,r_n$''; it is the smallest field extension of $F$ that contains $r_1,r_2,\ldots,r_n$.

In the following problems, we are working with subfields of $\mathbb{C}$, even if we don't say it explicitly. Thus, by Theorem~\ref{thm.BaseFieldC}, we are working with field extensions of $\mathbb{Q}.$

\begin{problem}\label{prob.QAdjoinI}
In Problem~\ref{prob.SubfieldRoot2PlusI}\ref{prob.SubfieldRoot2PlusI.QAdjoinI}, we saw that $\{a+bi\mid a,b\in \mathbb{Q}\}$ is a subfield of $\mathbb{C}$. 
Show that $\mathbb{Q}(i) = \{a+bi\mid a,b\in \mathbb{Q}\}$. 
\end{problem}

\begin{problem}
Show that $\mathbb{R}(i) =\mathbb{C}$. 
\end{problem}

\begin{problem}\label{prob.QAdjoinRoot5}
We saw previously that $\{a+b\sqrt{5}\mid a,b\in \mathbb{Q}\}$ is a subfield of $\mathbb{C}$. Find some $z\in \mathbb{C}$, such that $\mathbb{Q}(z) = \{a+b\sqrt{5}\mid a,b\in \mathbb{Q}\}$, and prove that your choice for $z$ works. Do you think there is only one choice for $z$ or might others work?
\end{problem}

\begin{problem}\label{prob.QAdjoinRoot2PlusI}
Let $\alpha = \sqrt{2} + i$. Show  $\{a+b\alpha\mid a,b\in \mathbb{Q}\} \subset \mathbb{Q}(\alpha)$, but $\{a+b\alpha\mid a,b\in \mathbb{Q}\} \neq \mathbb{Q}(\alpha)$. 
\end{problem}


\begin{theorem}\label{thm.FieldAdjoinElementsContainedInField}
Let $F$ be a subfield of $E$, and let $r_1,r_2,\ldots,r_n \in E$. If $K$ is any subfield of $E$, then $F(r_1,r_2,\ldots,r_n) \subseteq K$ if and only if $F\subseteq K$ and $r_1,r_2,\ldots,r_n\in K$.
\end{theorem}

\begin{problem}\label{prob.QAdjoinRoot27Root7}
Show that $\mathbb{Q}\left(3-\sqrt{2},5+i\right) = \mathbb{Q}\left(\sqrt{2},i\right)$.
\end{problem}

\begin{problem}\label{prob.DiagramSubfieldsRoot2andi}
Complete the diagram below to illustrate how each of the following sets intersect and where each element is located. Each set that is a field should be drawn in blue; each set that is not a field should be drawn in red. Elements should be illustrated by a dot and then labeled by the name of the element. Some have already been done.
\[\mathbb{C},\mathbb{R},\mathbb{Q}, \mathbb{Z}, 0, 1, \sqrt{2}, i, i\sqrt{2}, \sqrt{2}+i, \mathbb{Q}\left(\sqrt{2}\right), \mathbb{Q}(i), \mathbb{Q}\left(i\sqrt{2}\right) \mathbb{Q}\left(\sqrt{2},i\right), \{a+bi\mid a,b\in \mathbb{Z}\}\]
\begin{center}
\begin{tikzpicture}[line width = 1,xscale = .9, yscale = .5]
\draw[domain = 0:360,blue,samples = 100] plot ({5*cos(\x)},{4*sin(\x)});
\begin{scope}[shift={(-1,-0)}]
\draw[domain = 0:360,blue,samples = 100,rotate = 45] plot ({3.5*cos(\x)},{2.5*sin(\x)});
\end{scope}
\begin{scope}[shift={(0.5,-1.75)}]
\draw[domain = 0:360,blue,samples = 100,rotate = 10] plot ({3*cos(\x)},{1.7*sin(\x)});
\end{scope}
\begin{scope}[shift={(-0.6,-1.4)}]
\draw[domain = 0:360,red,samples = 100,rotate = 35] plot ({1.5*cos(\x)},{0.5*sin(\x)});
\end{scope}
\node[blue] at (3,2.5) {$\mathbb{C}$};
\node[blue] at (0.5,2) {$\mathbb{R}$};
\node[blue] at (2.7,-1.2) {$\mathbb{Q}(i)$};
\node[red] at (0.1,-0.9) {$\mathbb{Z}$};
\node[red] (root2) at (-2.6,-0.9) {$\sqrt{2}$};
\fill[red,anchor = 45,yscale = 10/7] (-2.9,-.9) circle (0.07);
\end{tikzpicture}
\end{center}
\end{problem}

\begin{problem}
Conjecture where $\mathbb{Q}\left(\sqrt{2}+i\right)$ would be in the previous diagram.
\end{problem}

Suppose that $F_1$ and $F_2$ are subfields of $E$. Theorem~\ref{thm.IntersectFields} tells us that $F_1\cap F_2$ is again a subfield, and it is the largest subfield contained in both $F_1$ and $F_2$. Then Definition~\ref{def.GenerateField}, also tells us that $\langle F_1\cup F_2 \rangle_{\textsc{field}}$ is a subfield, and it is the smallest subfield containing both $F_1$and $F_2$. This implies that the set of all subfields of $E$ forms a \href{https://en.wikipedia.org/wiki/Lattice_(order)}{lattice}. Lattices will not be defined here, but feel free to look them up. We will, however, be interested in illustrating these relationships with a diagram. The situation for $F_1$ and $F_2$ described above would be drawn as follows.
\begin{center}
\begin{tikzpicture}[line width = 1, scale = 1.5]
\node (E) at (0,0) {$E$};
\node (F1F2) at (0,-1)  {$\langle F_1\cup F_2 \rangle$};
\node (F1) at (-1,-2) {$F_1$};
\node (F2) at (1,-2) {$F_2$};
\node (F1capF2) at (0,-3) {$F_1\cap F_2$};
\draw (E) -- (F1F2);
\draw (F1F2) -- (F1);\draw (F1F2) -- (F2);
\draw (F1capF2) -- (F1);\draw (F1capF2) -- (F2);
\end{tikzpicture}
\end{center}

For a concrete example, let's draw the portion of the subfield lattice of $\mathbb{C}$ containing $\mathbb{Q}$, $\mathbb{R}$, $\mathbb{Q}(\sqrt{2})$, and $\mathbb{Q}\left(\sqrt{2},i\right)$; this uses some of what you discovered in Problem~\ref{prob.DiagramSubfieldsRoot2andi}.

\begin{center}
\begin{tikzpicture}[line width = 1, scale = 1.5]
\node (C) at (0,0) {$\mathbb{C}$};
\node (R) at (-1,-1)  {$\mathbb{R}$};
\node (Qi2) at (1,-1) {$\mathbb{Q}(\sqrt{2},i)$};
\node (Q2) at (0,-2) {$\mathbb{Q}(\sqrt{2})$};
\node (Q) at (0,-3) {$\,\mathbb{Q}$};
\draw (C) -- (R);\draw (C) -- (Qi2);
\draw (Qi2) -- (Q2);\draw (R) -- (Q2);
\draw (Q) -- (Q2);
\end{tikzpicture}
\end{center}

\begin{problem}\label{prob.LatticeSubfieldsRoot2andi}
Draw the portion of the subfield lattice of $\mathbb{C}$ that contains the following fields:
$\mathbb{C}$, $\mathbb{R}$, $\mathbb{Q}$, $\mathbb{Q}\left(\sqrt{2}\right)$, $\mathbb{Q}(i)$, $\mathbb{Q}\left(i\sqrt{2}\right)$, and $\mathbb{Q}\left(\sqrt{2},i\right)$.
\end{problem}

\begin{problem}
Draw the portion of the subfield lattice of $\mathbb{C}$ that contains the following fields:
$\mathbb{C}$, $\mathbb{Q}$, $\mathbb{Q}(\zeta_4)$, $\mathbb{Q}(\zeta_8)$, and $\mathbb{Q}(\zeta_{16})$.
\end{problem}


\appendix
% !TEX root = IBL-InsolvabilityOfQuintic.tex
\chapter{Hints}
\label{chapter:Hints}
\thispagestyle{empty}

Below are some hints, which should be interpreted as possible (but not the only!) ways to get started.

\begin{hint*}[Theorem~\ref{thm.MonicQuadratic}]
You are solving $x^2+bx+c = 0$. Try ``completing the square'' first; then solve for $x$.
\end{hint*}

\begin{hint*}[Problem~\ref{prob.ComplexCheckin}]
Multiplying a fraction by the complex conjugate of the denominator can be an effective way to simplify an expression.
\end{hint*}

\begin{hint*}[Theorem~\ref{thm.PolarToRectangular}]
Think back to changing from polar to rectangular coordinates (or parametrizing circles or solving triangles).
\end{hint*}

\begin{hint*}[Theorem~\ref{thm.MultiplyComplex}]
Try using Theorem~\ref{thm.PolarToRectangular} $+$ trigonometric identities. 
\end{hint*}

\begin{hint*}[Problem~\ref{prob.nthRoots}]
You want to find a $z$ such that $z^4 = \zeta_3$. You are working with powers (hence multiplication), so try writing $z$ in the form $z = r\cos\theta + ir\sin\theta$. Now you can use Corollary~\ref{cor.DeMoivre} to simplify $z^4$ and compare with $\zeta_3$. What can you deduce about $r$ and $\theta$?
\end{hint*}

\begin{hint*}[Lemma~\ref{lem.nthRoot1IsPowerOfZeta}]
Similar to Problem~\ref{prob.nthRoots}, try writing $z$ in the form $z = r\cos\theta + ir\sin\theta$. Now, what does $z^n = 1$ imply about $r$ and $\theta$?
\end{hint*}

\begin{hint*}[Lemma~\ref{lem.ReducePowerOfZeta}]
It may be helpful to draw some pictures first. Try plotting $\zeta_8$, $(\zeta_8)^2$, $(\zeta_8)^3$, \ldots, $(\zeta_8)^8$, $(\zeta_8)^{14}$, $(\zeta_8)^{85}$. Now, you know by a previous problem that $(\zeta_n)^n = 1$, so also $(\zeta_n)^{2n} = 1$ and so on. Try (using the division algorithm) to write $k = qn +r$ for some $q,r\in \mathbb{Z}$ with $0\le r \le n-1$ and plug that into $(\zeta_n)^{k}$.
\end{hint*}


\begin{hint*}[Theorem~\ref{thm.nthRoots1}]
You may want to view this as the following ``if and only if'' statement: $z$ is an $n^\text{th}$ root of $1$ $\iff$ $z = (\zeta_n)^k$ for some $0\le k\le n-1$. Now make use of the previous lemma and theorems you proved. Don't forget to explain why each of $1, \zeta_n, (\zeta_n)^2, \ldots, (\zeta_n)^{n-1}$ are all different.
\end{hint*}

\begin{hint*}[Theorem~\ref{thm.RootsRealCoeff}]
Suppose that $z$ is a root of $p(x)$. Then $p(z) = 0$, so  $a_nz^n + a_{n-1}z^{n-1} +\cdots+a_2z^2+a_1z+a_0 = 0$. This last equation is is just comparing two complex numbers---try taking the conjugate of both sides. Fact~\ref{fact.ComplexLaws} is helpful.
\end{hint*}

\begin{hint*}[Problem~\ref{prob.QAdjoinRoot5Inverse}]
You are trying to find $(a+b\sqrt{5})^{-1} = \frac{1}{a+b\sqrt{5}}$. Try multiplying top and bottom by the conjugate: $a-b\sqrt{5}$.
\end{hint*}

\begin{hint*}[Theorem~\ref{thm.BasicFieldProps}]
For the first part, notice that $x\cdot0 = x(0+0)$. For the last part, remember that the definition of a field ensures that $F$ has at least two elements, so there is some $a\in F$ with $a\neq 0$. Now, what happens if $0=1$?
\end{hint*}

\begin{hint*}[Theorem~\ref{thm.ZpField}]
The crux is to show that every nonzero element has a multiplicative inverse when $n$ is prime. Let $a\in (\mathbb{Z}_n)^*$. You need to find some integer $b$ such that $ab=1$ modulo $n$. Now, since $a\in (\mathbb{Z}_n)^*$ and  $n$ is prime, $\gcd(a,n) = 1$. By B\'ezout's Lemma, there exist $k,l\in \mathbb{Z}$ such that $1 = ka+ln$. What happens when you consider the equation $1 = ka+lp$ modulo $n$?
\end{hint*}

\begin{hint*}[Problem~\ref{prob.SubfieldRoot2PlusI}]
If $T_3$ is a subfield, then, in particular, it is closed under multiplication, so it must be that $\alpha^2\in T_3$. That means that $\alpha^2 = a+b\alpha$ for some $a,b\in \mathbb{Q}$. What does this imply?
\end{hint*}

\begin{hint*}[Problem~\ref{prob.QAdjoinI}]
Try following the approach in Example~\ref{exam.GenerateField}. First show  $\{a+bi\mid a,b\in \mathbb{Q}\} \subseteq \mathbb{Q}(i)$ by showing that every subfield that contains $\mathbb{Q}$ and $i$ must also contain $\{a+bi\mid a,b\in \mathbb{Q}\}$. To show the reverse containment, use the fact that $\{a+bi\mid a,b\in \mathbb{Q}\}$ is a subfield, by a previous problem.
\end{hint*}

\begin{hint*}[Problem~\ref{prob.QAdjoinRoot2PlusI}]
Remember, in Problem~\ref{prob.SubfieldRoot2PlusI}\ref{prob.SubfieldRoot2PlusI.QAdjoinRoot2PlusI}, we saw that $\{a+b\alpha\mid a,b\in \mathbb{Q}\}$ is \emph{not} a subfield of $\mathbb{C}$.
\end{hint*}

\begin{hint*}[Problem~\ref{prob.QAdjoinRoot27Root7}]
Use the previous theorem. To show  $\mathbb{Q}\left(3-\sqrt{2},5+i\right) \subseteq \mathbb{Q}\left(\sqrt{2},i\right)$, you need to show that $\mathbb{Q}\subset \mathbb{Q}\left(\sqrt{2},i\right)$ and that $3-\sqrt{2},5+i\in \mathbb{Q}\left(\sqrt{2},i\right)$. Then show the reverse containment in a similar way.
\end{hint*}

\begin{hint*}[Theorem~\ref{thm.SolvableByRadicalsNontrivialRootsOf1}]
Note that $x^n - 1 = (x-1)(x^{n-1} + x^{n-2} + \cdots + x^2 + x + 1)$. Now use Theorem~\ref{thm.nthRoots1}; note that $x^{n-1} + x^{n-2} + \cdots + x^2 + x + 1$ should only have  $n-1$ roots.
\end{hint*}

\begin{hint*}[Problem~\ref{prob.SolvableByRadicalsHard}]
First find the roots of $z^2 - 3z - 1$. Then, for each of those roots, use Theorem~\ref{thm.nthRoots} to solve for $z$. You should have 6 different roots in the end.
\end{hint*}

\begin{hint*}[Theorem~\ref{thm.UnitIsNotZeroDivisor}]
Try a proof by contradiction. Assume that $u$ is a unit and that $u$ is a zero divisor. Now, what does the definition of being a zero divisor tell you about $u$?
\end{hint*}

\begin{hint*}[Theorem~\ref{thm.DegreePolySum}]
To get started, let $n = \deg p(x)$ and $m = \deg q(x)$, and then write $p(x) = a_0 + a_1x + \cdots + a_nx^n$ with $a_n\neq 0$ and $q(x) = b_0 + b_1x + \cdots + b_mx^m$ with $b_m\neq 0$. You want to understand the degree of $p(x) + q(x)$, so you need to determine the largest power of $x$ in the sum $p(x) + q(x)$.
\end{hint*}

\begin{hint*}[Theorem~\ref{thm.DegreePolyProduct}]
As with the previous theorem, let $n = \deg p(x)$ and $m = \deg q(x)$, and then write $p(x) = a_0 + a_1x + \cdots + a_nx^n$ with $a_n\neq 0$ and $q(x) = b_0 + b_1x + \cdots + b_mx^m$ with $b_m\neq 0$. You need to determine the largest power of $x$ in the product $p(x)q(x)$. What do you think is the largest power of $x$ in the product $p(x)q(x)$? What is its coefficient, and how do you know it's not zero?
\end{hint*}

\begin{hint*}[Corollary~\ref{cor.PolysOverIntegralDomains}]
There are several things to verify to ensure that $D[x]$ is an integral domain, but we've talked about most of them already. The main thing that remains is to prove that $D[x]$ has no zero divisors---try a proof by contradiction. This is a corollary of Theorem~\ref{thm.DegreePolyProduct}, which means that it should be ``not too hard'' to prove using Theorem~\ref{thm.DegreePolyProduct}. 
\end{hint*}

\begin{hint*}[Theorem~\ref{thm.DivisionAlgorithm}]
One approach is to polish up and fill in the gaps of the outline presented in the notes right before the statement of Theorem~\ref{thm.DivisionAlgorithm}. A related, but slightly different, approach is to try using induction on the degree of $a(x)$.
\end{hint*}

\begin{hint*}[Theorem~\ref{thm.RootImpliesLinearFactorOfPoly}]
Try  using the division algorithm to write $a(x) = (x-c)q(x) + r(x)$ for some $q(x),r(x)\in F[x]$ with $\deg r(x) < \deg (x-c)$ or $r(x) = 0$. Now show that $r(x)$ must be the zero polynomial.
\end{hint*}

\begin{hint*}[Lemma~\ref{lem.GCDUnique}]
First, explain why $d_1(x)$ must divide $d_2(x)$ and why $d_2(x)$ must divide $d_1(x)$. Now return to the definition of ``to divide'' and see what you can write down.
\end{hint*}

\begin{hint*}[Theorem~\ref{thm.HalfOfGCDProof}]
Follow the definitions. Since $c(x)\in I$, it can be written a particular way. Then write down what it means for $h(x)$ to divide both $a(x)$ and $b(x)$. Combine.
\end{hint*}

\begin{hint*}[Theorem~\ref{thm.UnitsFAdjoinx}]
For the forward direction, start with the definition of a unit and apply the degree function. For the reverse direction, what does $\deg p(x) = 0$ imply about $p(x)$? Can you explicitly write down the a multiplicative inverse for $p(x)$?
\end{hint*}

\begin{hint*}[Theorem~\ref{thm.ReducibilityTestDegree2or3}]
Consider using Theorem~\ref{thm.LinearFactorOfPolyImpliesRoot}.
\end{hint*}

\begin{hint*}[Theorem~\ref{thm.FactorIrreducibles}]
Consider using using strong induction on the degree of the polynomial. Let $\varphi(n)$ be the statement ``every polynomial in $F[x]$ of degree $n$ can be written as a product of polynomials that are irreducible in $F[x]$.'' 

For the base case, you want to show that $\varphi(1)$ is true. Assume that $p(x)\in F[x]$ has degree $1$. Then what? 

Next, assume that $\varphi(k)$ is true for all $1\le k \le n$. We need to show that $\varphi(n+1)$ is true. Assume that $p(x)\in F[x]$ has degree $n+1$. There are two cases to consider: $p(x)$ is irreducible or $p(x)$ is reducible. Keep going\ldots
\end{hint*}









\end{document}