\chapter{Solving polynomial equations}
\label{chapter:PolyEquations}
\thispagestyle{empty}



Can you count all of the times in your career that you've had to find the zeros of a quadratic polynomial? What about a cubic polynomial? A quartic? Likely your answers are decreasing rapidly, and it's also likely that you have only solved cubic and quartic in very special situations. Why is this? Is it just that cubic and quartic equations are difficult to solve or could it be that some are impossible to ``solve.'' Let's explore this briefly.

\begin{exercise}
Determine the roots (i.e.\ zeros) of each of the following. Try to use tools you've accumulated over the years, but you can also use a computer program (e.g. \href{https://www.wolframalpha.com}{WolframAlpha}) if you need. For each problem, make a note about \emph{how} you found the roots.
\begin{multicols}{2}
\begin{enumerate}
\item $p(x) = x^2 - 5x + 6$
\item $q(x) = x^2 - 2$
\item $r(x) = x^2 + x + 1$
\item $s(x) = x^3 - 4x^2 + 1x + 6$
\item $f(x) = x^4 - 5x^2 + 6$
\item $g(x) = x^5 - 1$
\end{enumerate}
\end{multicols}
\end{exercise}

\begin{section}{Solving polynomial equations with formulas}
Let's start by (re)developing everyone's favorite formula. We'll do it in stages.



\end{section}
