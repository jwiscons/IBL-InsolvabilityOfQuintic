% !TEX root = IBL-InsolvabilityOfQuintic.tex
\chapter{Solving polynomial equations}
\label{chapter:PolyEquations}
\thispagestyle{empty}



Can you count all of the times in your career that you've had to find the zeros of a quadratic polynomial? What about a cubic polynomial? A quartic? Likely your answers are decreasing rapidly, and it's also likely that you have only solved cubic and quartic in very special situations. Why is this? Is it just that cubic and quartic equations are difficult to solve or could it be that some are impossible to ``solve.'' Let's explore this briefly.

\begin{problem}\label{prob:IntroFindZeros}
Determine the roots (i.e.\ zeros) of each of the following. Try to use tools you've accumulated over the years, but you can also use a computer program (e.g. \href{https://www.wolframalpha.com}{WolframAlpha}) if you need. For each problem, make a note about \emph{how} you found the roots. Try for exact answers, but you can approximate if needed.
\begin{multicols}{2}
\begin{enumerate}
\item $p(x) = x^2 - 5x + 6$
\item $q(x) = (x-3)^2 - 2$
\item $r(x) = x^2 + x + 1$
\item $s(x) = x^3 - 4x^2 + 1x + 6$ %rational root theorem
\item $f(x) = x^4 - 1$
\item $g(x) = x^4 - 5x^2 + 6$
\item $a(x) = x^5 - 1$
\item $b(x) = x^5 +5 x^4-5$
\end{enumerate}
\end{multicols}
\end{problem}

\begin{problem}\label{prob:IntroFindZerosWhere}
For each part of Problem~\ref{prob:IntroFindZeros}, write down the ``smallest'' number system needed to express the roots of the given polynomial. Possible answers might be $\mathbb{Z}$ (integers), $\mathbb{Q}$ (rational numbers), $\mathbb{Z}$ together with $\sqrt{3}$, $\mathbb{Q}$ together with $i=\sqrt{-1}$, etc.
\end{problem}

\section{Solving polynomial equations with formulas}
Though you may have solved the first three parts of Problem~\ref{prob:IntroFindZeros} different ways, there was one tool that would have solved them all: the quadratic formula. It will be valuable to (re)discover why it's true. 

First, let's slightly simplify things. Notice that $\alpha$ is a root of $ax^2 + bx + c$ if and only if $\alpha$ is a root of $x^2 + \frac{b}{a}x + \frac{c}{a}$ (assuming $a\neq 0$). This means that an arbitrary quadratic polynomial can always be convert to a quadratic polynomial whose leading coefficient is $1$ in such a way that they have the same roots. Thus, if we have a formula that finds the roots of so-called \emph{monic} quadratic polynomials, we can actually use it to find the roots of \emph{all} quadratic polynomials.

\begin{definition}
A polynomial whose leading coefficient is $1$ is called a $\textbf{monic}$ polynomial.
\end{definition}

Now, let's (re)derive the quadratic formula for monic polynomials. Remember, we are \emph{(re)deriving} it, so \emph{please don't use the quadratic formula in your proof of the next theorem.}

\begin{theorem}\label{thm:MonicQuadratic}
If $p(x) = x^2 + bx + c$, then the roots of $p(x) = x^2 + bx + c$ are \[\frac{-b\pm\sqrt{b^2 - 4c}}{2}.\]
\end{theorem}

\begin{problem}
Write down the ``smallest'' number system needed to express the roots of \emph{any} quadratic polynomial.
\end{problem}

Well, that takes care of quadratic polynomials. What about finding the roots of cubic polynomials?



