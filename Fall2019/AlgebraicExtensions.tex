% !TEX root = IBL-InsolvabilityOfQuintic.tex
\chapter{Algebraic extension fields}
\label{chapter:AlgebraicExtensions}
\thispagestyle{empty}


Our goal remains: to show that there exist polynomials that are \emph{not} solvable by radicals over $\mathbb{Q}$. In Chapter~\ref{chapter:SolvabilityByRadicals}, we finally succeeded in properly formulating the notion of solvability by radicals, which we did in the language of field extensions. Additionally, we were able to catalog many polynomials that we are certain are solvable by radicals. Then, in Chapter~\ref{chapter:Rings}, we took a much closer look a polynomials, ultimately building a significant amount of language and theory to analyze polynomial rings over fields. The conclusion of the chapter hinted at the tight connection between polynomial rings and field extensions where we saw that $\mathbb{Q}[x]/(x^2+1) \cong \mathbb{Q}(i)$. In this chapter, we will clarify this connection and exploit it to significantly deepen our understanding of extension fields of the form $\mathbb{Q}(\alpha)$ where $\alpha$ is a root of some polynomial in $\mathbb{Q}[x]$. 


% % % % % % % % % % % % % % % % % % % % % % % % % % % % % % % % % % % % % % % % % % % %
% % % % % % % % % % % % % % % % % % % % % % % % % % % % % % % % % % % % % % % % % % % %
% SECTION
% % % % % % % % % % % % % % % % % % % % % % % % % % % % % % % % % % % % % % % % % % % %
% % % % % % % % % % % % % % % % % % % % % % % % % % % % % % % % % % % % % % % % % % % %
\section{Algebraic elements}

Recall that a polynomial in $F[x]$ is solvable by radicals over $F$ if all of the roots of the polynomial lie in some radical extension of $F$. Our focus is on roots of polynomials, and the next definition gives us some language to highlight this.

\begin{definition}
Let $F$ be a subfield of $E$. An element $\alpha\in E$ is \textbf{algebraic} over $F$ if $p(\alpha) = 0$ for some nonzero $p(x)\in F[x]$. If $\alpha$ is not algebraic over  $F$, then it is said to be \textbf{transendental} over $F$. 
\end{definition}

To show an element $\alpha$ is algebraic over $F$, we need only produce a polynomial \emph{with coefficients in $F$} for which $\alpha$ is a root. For example, the complex number $\sqrt{2}$ is algebraic over $\mathbb{Q}$ because $\sqrt{2}$ is a root of $p(x) = x^2 - 2$ and $p(x) \in \mathbb{Q}[x]$. Also, $\pi$ is algebraic over $\mathbb{R}$ because $\pi$ is a root of $q(x) = x-\pi$ and $q(x) \in \mathbb{R}[x]$. However, it is \emph{much} harder to show that $\pi$ is \emph{not} algebraic over $\mathbb{Q}$ (so  $\pi$ is transcendental over $\mathbb{Q}$). Incidentally, a set-theoretic argument  shows that almost all elements of $\mathbb{C}$ are transcendental over $\mathbb{Q}$; nevertheless, we will focus on algebraic elements.

\begin{problem}
Show that each of the following are algebraic over $\mathbb{Q}$: $11$, $\sqrt[3]{11}$, $\zeta_{11}$, and $i$.
\end{problem}

\begin{problem}
Suppose that $\gamma \in \mathbb{C}$ and $\gamma^5 = 2\gamma^2 -7$. Explain why $\gamma$ is algebraic over $\mathbb{Q}$.
\end{problem}

\begin{problem}\label{prob.Sqrt2PlusIIsAlgebraic}
Let $\alpha = \sqrt{2} + i$. Show that $\alpha$ is algebraic over $\mathbb{Q}$.
\end{problem}

Now, an algebraic element over $F$ is a root of \emph{some} nonzero polynomial over $F$, but such an element will be a root of lots of polynomials. For example, since $\sqrt{2}$ is a root of $p(x) = x^2 - 2$, we find that for \emph{every} $q(x)\in \mathbb{Q}[x]$, $\sqrt{2}$ is a root of $q(x)p(x)$ because $q(\sqrt{2})p(\sqrt{2}) =  q(\sqrt{2})p(0) =0$. The polynomial $x^2 - 2$ is special because it is a polynomial \emph{of smallest degree} for which $\sqrt{2}$ is a root. In order to formalize this observation, we need to weave together several results from the last chapter. 

Let $F$ be a subfield of $E$, and suppose that $\alpha\in E$ is algebraic over $F$. Then $\alpha$ is the root of some nonzero $p(x)\in F[x]$.  Let's look at the set of all polynomials for which $\alpha$ is a root: $I = \{p(x)\in F[x]\mid p(\alpha) = 0\}$. By Theorem~\ref{thm.EvalHom}, ``evaluation at $\alpha$'' gives rise to a homomorphism $\mathcal{E}_\alpha: F[x] \to E[x]$, and notice that $I$ is precisely the kernel of $\mathcal{E}_\alpha$. Thus, by Theorem~\ref{thm.KernelIsIdealImageIsSubring}, $I$ is an ideal of $F[x]$, and since $F[x]$ is a PID, it must be that $I = (m(x))$ for some $m(x) \in F[x]$. 

Now, $m(x)$ is nonzero since $I$ contains some nonzero polynomial, and this also implies that $m(x)$ is not a constant polynomial since the only constant polynomial that has roots is the zero polynomial. Moreover, if $m(x)$ is not monic, we can make it monic by multiplying by the inverse of the leading coefficient and the result will still generate $I$ by Theorem~\ref{thm.DifferByUnitSameIdeal}. So, we may assume that $I = (m(x))$ with $m(x)$ nonconstant and monic. 

Also, notice that by Theorem~\ref{thm.PrincipalIdeals}, $m(x)$ divides every polynomial in $I$. So if $I = (n(x))$ for some other monic polynomial $n(x)$, then $m(x)$ and $n(x)$ would divide each other. In other words, $m(x) = a(x)n(x) = a(x)b(x)m(x)$ for some polynomials $a(x),b(x)\in F[x]$. Considering the degree of both sides of $m(x) = a(x)b(x)m(x)$, we see that $a(x)$ and $b(x)$ must be constant polynomials. But since $m(x) = a(x)n(x)$ with $m(x)$ and $n(x)$ both monic, the only conclusion is that $a(x) = 1$, so $m(x) = n(x)$. 

In summary, $I= \{p(x)\in F[x]\mid p(\alpha) = 0\}$ is an ideal, and there is a unique nonconstant monic polynomial $m(x)$ that generates $I$. In fact, more is true. 

\begin{lemma}\label{lem.MinimalPolyIsIrreducible}
Let $F$ be a subfield of $E$, and suppose that $\alpha\in E$ is algebraic over $F$. Let $I = \{p(x)\in F[x]\mid p(\alpha) = 0\}$, and suppose that $I = (m(x))$ for some nonconstant $m(x) \in F[x]$. Then $m(x)$ is irreducible.
\end{lemma}

Combining Lemma~\ref{lem.MinimalPolyIsIrreducible} with our previous discussion, we arrive at the following fact.

\begin{fact}\label{fact.ExistsMinimalPoly}
Let $F$ be a subfield of $E$, and suppose that $\alpha\in E$ is algebraic over $F$. Then there is a unique irreducible monic polynomial $m(x) \in F[x]$ such that $m(r) = 0$. Moreover, if $p(x) \in F[x]$ and $p(\alpha) = 0$, then $m(x)$ divides $p(x)$.
\end{fact}



%\begin{problem}
%Recall that $\mathbb{Q}\left(\sqrt{5}\right)$ can be written as $\mathbb{Q}\left(\sqrt{5}\right) = \left\{a+b\sqrt{5}\mid a,b\in \mathbb{Q}\right\}$. Show that $\phi:\mathbb{Q}\left(\sqrt{5}\right) \rightarrow \mathbb{Q}\left(\sqrt{5}\right)$ defined by $\phi(a+b\sqrt{5}) = a-b\sqrt{5}$ is an isomorphism.
%\end{problem}